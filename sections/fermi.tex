% !TeX root = ../main.tex

\subsection{Fermi coordinates}\label{sec:fermi}

Our framework is a specific case of the one used in~\cite{Pacard2014} for a
1-dimensional manifold, and we will indeed follow their notations. Consider the
tubular neighbourhood $T_\ep$ as defined in~\ref{eq-tube}; The Fermi coordinates
parameterise this set as a product space between the curve $\Lambda$ and
$B^{n-1}$ as follows. First identify $\Lambda$ with the zero-section of
$N\Lambda$ (the normal bundle of $\Lambda$) and $T_\ep$ with
\[
\Omega_\ep:=\{(y,z)\in N\Lambda;\, y\in\Omega,\, z\in N_y\Lambda,\, |z|\leq\ep\}
\]
via the natural mapping $T_\ep\to\Omega_\ep$, $(y,z)\mapsto y+z$.

If $g_z:=dz^2$ is the Euclidean metric on normal fibers, and $\mathring{g}$ is
the metric induced on $\Lambda$, we have that the metric $\bar g$ on $N\Lambda$
is induced by the embedding of $\Lambda$ in $\RR^n$, this is $\bar g=\mathring
g+g_z$. Lemma 3.2 of~\cite{Pacard2014} proves that, in these coordinates, the
Euclidean Laplacian $\Delta$ can be decomposed as
\begin{equation}
\Delta = \Delta_{\bar g} + D,
\end{equation}
where $\Delta_{\bar g}=\Delta_{\mathring g}+\Delta_{g_z}$ denotes the
Laplace-Beltrami operator on $N\Lambda$ for metric $\bar g$, and $D$ is a
second-order differential operator of the form
\begin{equation}
    D = \sum_{i=1}^{n-1} z_{i}D_i^{(2)}+D^{(1)},
\end{equation}
where $D^{(1)}$ and $D^{(2)}$ are first-order and second-order partial
differential operators respectively, whose
coefficients are smooth and bounded.
Note that in this $1$--dimensional manifold, we have a parameterisation
$t:\RR\to\Lambda$ of the curve, and the Laplace-Beltrami operator of $\Lambda$
is simply $\partial_{tt}$. We therefore have
\begin{equation}
    \Delta = \partial_{tt}+\Delta_{g_z}+D.
\end{equation}
This decomposition, and in particular the form of the operator $D$, will allow
us to obtain estimates for functions defined in small tubes and prove our main
results, which we describe in the next sections.
