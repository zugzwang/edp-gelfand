% !TeX root = ../main.tex

\subsection{The case $\Omega=B^n$}\label{sec:ball}
The celebrated theorem of Gidas-Ni-Nirenberg (see~\cite{gidas-ni-nirenberg})
establishes that the solutions of
\begin{equation}
    (\mbox{GNN})\;\left\{
        \begin{array}{cc} \Delta u + f(u)=0 & \mbox{ in }B^n,\\
        u=0 & \mbox{ on }\partial B^n,
    \end{array}\right.
\end{equation}
are positive and radially symmetric provided that $f$ is positive and regular,
where $B^n=\{x\in\RR^n,|x|\leq 1\}$. Therefore, if $\Omega=B^n$, all solutions
of $(G_\lambda)$ are radially symmetric and $(G_\lambda)$ is equivalent to the
problem of finding $u:[0,1]\to\RR$ such that
\begin{equation}
    \left\{\begin{array}{lc}
            u''+\frac{n-1}{r}u'+\lambda\, e^u =0, & r\in (0,1),\\
    u'(0) = u(1) = 0.&  \end{array}\right.
\end{equation}
Notice that $||u||_{L^\infty}=u(0)$. For $n=1$, the equation $u''+\lambda e^u=0$
may be explicitely solved by means of the Laplace transform (see~\cite{bratu}),
giving $u(x)=2\,\log\big(\alpha\,\mbox{sech}(\alpha\sqrt{\lambda/2}\cdot
x)\big)$, and the boundary conditions require that $\alpha$ is a solution to
the transcendental equation $\alpha
=\mbox{cosh}\big(\alpha\sqrt{\lambda/2}\big)$. One verifies that this leads to
zero, one, or two solutions for $\lambda>\lambda_c,\lambda=\lambda_c$ and
$\lambda<\lambda_c$ respectively, where $\lambda_c\approx.88$. Observe that
$||u||_{L^\infty}=u(0)=2\,\log\alpha$, and from the transcendental equation one
computes $\alpha_1=1+\frac{\lambda}{4}+o(1)$, $\alpha_2=\frac{4}{\lambda}+o(1)$.
It thus follows that, for any small and fixed $\lambda$, one solution
approaches to $||u||=0$ and the other one approaches to $||u||=\infty$, this is,
the solution curve $(\lambda, ||u_\lambda||)$ is unbounded, contained in
$[0,\lambda_c]\times \RR^+$. One can easily check that, in the upper branch of
solutions, $\lambda$ decreases with $||u||$, yielding the nose-shape of the
solution curve.

Other branches of solutions have been computed by numerical means, see for
instance~\cite{new-solutions}.  The bi-dimensional case $\Omega=B^2$ can also
be solved explicitly and it presents a similar behaviour than the previous one.
Solutions exist if and only if $0\leq \lambda \leq 2$, and for $\lambda=2$
there is only one solution given by $u_\ast(r)=\log\frac{4}{{(1+r^2)}^2}$.
For other admissible values of $\lambda$, solutions are given by
\[
    u_i(r)=\log\frac{b_i}{{(1+(\lambda\, b_i/8)r^2)}^2},
\]
where
$b_i=\frac{32}{\lambda^2}\big(1-\frac{\lambda}{4}+{(-1)}^i\sqrt{1-\lambda/2}\big),
i=1,2$.
\medskip

There is, however, a remarkable difference between $n=1$ and $n=2$: For
$\Omega=(-1,1)$, the unstable solution blows up at every point as $\lambda\to
0$, whereas for $\Omega=B^2$, it blows up only at the origin.
\medskip

For general dimension $n\geq 3$, the problem is analysed with a suitable change
of variables, and the behaviour of the solution yields from a dynamical coupled
system that arises from the study of the radial equation. This is the approach
considered by Dupaigne in~\cite{stable-solutions-elliptic}; for other
instances of the Gelfand problem we strongly recommend this text. We summarize
the results for the radially symmetric case in every dimension in Table~\ref{table:bifurcation-ball}.

\begin{table}[ht!]\label{table:bifurcation-ball}
    \center\renewcommand{\arraystretch}{1.2}
    \begin{tabular}{c|p{.4\textwidth}|c}
        \midrule
        \textit{Dimension}                 & \textit{Number of solutions}                                            & \textit{Maximum value of $\lambda$}                  \\ \midrule
        \multirow{2}{*}{$n=1,2$}           & Two if $\lambda\in(0,\lambda_\ast)$                                     & \multirow{8}{*}{$\lambda_\ast(B^n)>2(n-2)$} \\ \cline{2-2}
                                           & One if $\lambda=\lambda_\ast$                                           &                                             \\ \cline{1-2}
        \multirow{6}{*}{$3\leq n  \leq 9$} & One if $\lambda$ is sufficiently small                                  &                                             \\ \cline{2-2}
                                           & Finitely many if $\lambda\neq 2(n-2)$                                   &                                             \\ \cline{2-2}
                                           & More than any given number for $\lambda$ sufficiently close to $2(n-2)$ &                                             \\ \cline{2-2}
                                           & Infinitely many for $\lambda = 2(n-2)$                                  &                                             \\ \cline{2-2}
                                           & Two for $\lambda$ close to $\lambda_\ast(B^n)$                          &                                             \\ \cline{2-2}
                                           & One for $\lambda = \lambda_\ast(B^n)$                                   &                                             \\ \midrule
        $n\geq 10$                         & One unique stable solution                                              & $\lambda_\ast(B^n)=2(n-2)$                  \\ \midrule
    \end{tabular}
    \caption{Bifurcation diagram for the Gelfand problem in $\Omega = B^n$ (\cite{stable-solutions-elliptic})}
\end{table}
