
\subsection{Exploiting Pacard-Pacella-Sciunzi's results}
\label{sec:exploiting-pacard}

In section 5 of \cite{Pacard2014}, authors prove that any operator of the form
$-(\Delta+V(x))$ is invertible in the tube, starting from the invertibility of
$-(\frac{1}{\ep^2}\Delta_{g_z}+\Delta_{\mathring g}+V(x))$ in the same domain.
Their analysis for the power non-linearity can be applied by analogy with no
further complications to obtain the invertibility of $\tilde L_\ep$.

Let us describe the decomposition of eigenfunctions of $\tilde L_\ep$. Let
$\Phi_1:B^{n-1}\to \RR$ denote the first eigenfunction of $-(\Delta+\lambda
e^{\tilde U})$ in the ball $B^{n-1}$ with Dirichlet boundary conditions. Using
separation of variables we have that the eigenfunction of $\hat L_\ep$
associated to the eigenvalue $\mu_1+\ep^2\nu_j$ can be decomposed as
$\phi_1(z)\psi_j(y)$, where $\psi_j$ is the $j$-th eigenfunction of
$-\ep^2\Delta_{\mathring g}$ in $\Lambda$ and $\phi_1(z):=\Phi_1(z/|\ep|)$. Here
we prove a similar property for the operator $\tilde L_\ep$. We insist on the
fact that the ideas and lemmas here are a direct adaptation of the power
non-linearity treated in \ref{sec:exploiting-pacard}. First, for fixed $\ep$
define the smooth function $a$ and the operator $\tilde H_\ep$ acting on
functions in the tube by
\begin{equation}
    \operatorname{dvol}_g = \operatorname{dvol}_{\bar g}\quad \mbox{ and }
    \quad \tilde H_\ep:= a\tilde L_\ep,
\end{equation}
so that $\tilde L_\ep$ is self-adjoint with respect to $L^2(T_\ep(\Lambda),g)$
and $\tilde H_\ep$ is self-adjoint with respect to $L^2(T_\ep(\Lambda),\bar
g)$. We then have the following.

\begin{lemma}
    Let $v$ be an eigenfunction of $\tilde H_\ep$ associated to the eigenvalue
    $\gamma$, and for a function $\psi$ defined on $\Lambda$ write
    \begin{equation}
        v(y,z):=\phi_1(z)\psi(y)+w(y,z)
    \end{equation}
    with the orthogonality condition
    \begin{equation}
        \forall h \in L^2(\Lambda),\; \int_{T_\ep}w\cdot \phi_1\cdot h\, \operatorname{dvol}_{\bar g} = 0.
    \end{equation}
    Then there exists a constant $C$ such that 
    \begin{equation}
        \int_{T_\ep}(|\nabla_{\bar g}w|^2+w^2)\operatorname{dvol}_{\bar g} \leq
        C\gamma\ep\int_{T_\ep} v^2\operatorname{dvol}_{\bar g}.
    \end{equation}
\end{lemma}
We omit the proof as it follows by analogy from the proof of Lemma 5.2 in
\cite{Pacard2014}. Remark, however, that this lemma is more accurate than its
analogue, because we do not need assumptions on the eigenvalue $\gamma$ and
because the estimation is given in term of a positive power of $\ep$. In
particular, we can see the function $w$ as a small perturbation of $v$ in the
$L^2(T_\ep(\Lambda))$ sense.
\medskip

Pacard \etal also give an estimate for the change rate of the eigenvalues as
$\ep$ increases, which we adapt here as well:

\begin{lemma}
    \label{lem-partial-nu}
    If $\nu$ is an eigenvalue of $\tilde H_\ep$ such that $\nu<\alpha$, then
    there is a constant $C$ such that
    $$\frac{\partial \nu}{\partial \ep} \geq C\alpha\ep.$$
\end{lemma}
\begin{proof}
    {\color{red} To do. Paragraph of page 14 is suspicious.}
\end{proof}
