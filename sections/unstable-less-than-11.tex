\section{Unstable solutions in $\RR^4,\RR^5,\dots,\RR^{10}$}

We prove the following theorem, stating that the bifurcation diagram of the
Gelfand problem in the tube is similar to the one given by
\cite{stable-solutions-elliptic}, provided $\ep$ is away from a resonant
set.

\begin{theorem}
    Let $4\leq n \leq 10$. There is a set $E$ that accumulates around 0, such
    that if $\ep\in E$, the following holds: 
    \begin{itemize}
        \item For small $\lambda$ there is a unique, stable solution to
           ~\ref{eq-thm-2-sols}.
        \item For $\lambda$ near $\lambda^\ast(B^{n-1})$, there are two
            solutions to~\ref{eq-thm-2-sols}, a stable and an unstable one.
        \item For any given $k\in \NN$, there exists $\lambda$ near $2(n-3)$
            such that the number of solutions of~\ref{eq-thm-2-sols} is at least
            $k$. The Morse index of each of the unstable solutions converges to
            $\infty$ as $\ep\to 0$ in $E$.
    \end{itemize}
\end{theorem}

{\color{red} too informal...}
