\section{Unstable solutions in $\RR^2$ and $\RR^3$}

In this section, we show that the preceding construction generalises to obtain
new solutions to the Liouville problem in the tube given an unstable solution to
the problem in the ball $B^{n-1}$. We know that, for $n\geq 11$, there is a
unique solution to the Liouville problem in $B^{n-1}$, which is stable, thus we
will not be led to new solutions. We therefore work in tubes embedded in
$\RR^n$ with $n=2,3$.

The key ingredient that allowed us to prove the existence of the stable solution
in the tube is the invertibility of the operator $L_\ep=-(\Delta_{\mathring
g}+\lambda e^{u_\ep})$, where we recall that $u_\ep$ is a copy of the solution
given by the Gidas-Ni-Nirenberg theorem in the unit ball. Assume $2\leq n\leq 3$
and $\Omega=B^{n-1}$. We know from Section~\ref{sec:ball} that, for every
$\lambda < \lambda^\ast(B^{n-1})$, there is only one solution other than the
stable one, which we note by $\tilde U$. In other words, $\tilde U$ solves
\begin{equation}
\label{eq:u-tilde}
\left\{
\begin{array}{cc}
\Delta \tilde U + \lambda e^{\tilde U}=0 & \mbox{ in }B^{n-1},\\
\tilde U = 0 &\mbox{ on }\partial B^{n-1}.
\end{array}
\right.
\end{equation}

It is well-known that
$$
\mu_1(-\Delta-\lambda^\ast e^{u_\ast},B^{n-1})=0,
$$
where $u_\ast$ is the solution to $(L_\lambda^\ast)$ in the ball and $\mu_1$
stands for the least eigenvalue (see~\cite{stable-solutions-elliptic}, \S 2),
is the only case where the operator $-\Delta-\lambda^\ast e^{u_\ast}$ is
non-invertible for these dimensions. In this case, let us note, for any
$\lambda < \lambda^\ast$,
$$\mu_1<0<\mu_2\leq \mu_3\leq \cdots$$
the eigenvalues of the operator $-\Delta-\lambda e^{\tilde U}$ in the unit ball
$B^{n-1}$, and define as before $\tilde u_\ep:T_\ep\to\RR$ by $\tilde
u_\ep(y,z)=\tilde U\big(\frac{\operatorname{dist}(z,\Lambda)}{\ep}\big)$. The
negative eigenvalue $\mu_1$ is the source of a resonance phenomenon, forcing
to tweak part of the argument. Let us explain briefly: Note
$$0=\nu_1<\nu_2\leq\nu_3\leq\dots $$
the eigenvalues of $-\Delta_{\mathring g}$ on $\Lambda$. If we proceed as
before, the argument fails when trying to invert $\tilde
L_\ep:=-\ep^2\Delta+\lambda e^{\tilde u_\ep}$. Indeed, with a rescaling to work
in $T_1( \Lambda )$, this operator becomes $(-\Delta_{g_z}-\ep^2
\Delta_{\mathring g}-\ep D):=\hat L - \ep D$. Now, the eigenvalues of $\hat L$
are of the form $\mu_i+\ep^2\nu_j$, and as $\mu<0$, there exist values of $\ep$
such that $\hat L$ is non invertible. Let us note
$$
S:=\{\ep>0\,:\; \mu_1+\ep^2\nu_j=0,\, j\in \NN\}
$$
the set of such values of $\ep$. Even if we restrict $\ep\not\in S$, we will not
conclude at once the invertibility of $\tilde L_\ep$, because the fixed-point
theorem uses the fact that the norm of the inverse is controlled by
$1/\delta_\ep$, where $\delta_\ep$ is the distance from $0$ to the spectrum,
which was $\mu_1$ before. Here, we only have the bound
$$
\delta_\ep\leq\ep^2 \max_{1\leq j \leq k}(\nu_{j+1}-\nu_j),
$$
where $k$ is the index of the first eigenvalue of the spectrum
$(\nu_j)_{j\in\NN}$ such that $\mu_1+\ep^2\nu_k>0$. In particular, we will not
find a contraction mapping leading to the invertibility of $\tilde L_\ep$.

Let us first prove a partial result, and concentrate in $\tilde L_\ep$ afterwards.

\begin{theorem}
    Let $n=2,3$. For small enough $\ep$ and such that $\tilde L_\ep$ is
    invertible, there exist two solutions to
    \begin{equation}
        \label{eq-thm-2-sols}
        \left\{
            \begin{array}{cc}
                \Delta u + \frac{\lambda}{\ep^2} e^u = 0 & \mbox{ in
                }T_\ep(\Lambda),\\
                    u=0 & \mbox{ on }\partial T_\ep(\Lambda).
                \end{array}
                \right.
    \end{equation}
    \end{theorem}
    \textit{Proof: } Define by analogy the functions $\tilde W, \tilde w_\ep$
    as supersolutions and prove the same estimates as before. The fact that
    $\tilde L_\ep$ is invertible ensures that we have elliptic estimates for
    $\tilde u_\ep, \tilde w_\ep$. To finish the construction, restate the
    problem~\ref{eq-thm-2-sols} as a fixed-point problem that can be easily
    solved as in the proof of theorem~\ref{thm:unique-stable-sol}, completing
    the proof.\hfill $\blacksquare$

    % !TeX root = ../main.tex

\subsection{Morse index of $\tilde L_\ep$}

The spectral analysis of $\tilde L_\ep$ follows directly from the ideas
of~\cite{Pacard2014}, where authors deal with a power non-linearity. We adapt
the same estimates here. Recall that the normal bundle $N\Lambda$ is endowed
with $\bar g= \mathring g + g_z$ in $\RR^n$, where $\mathring g$ is the induced
metric on $\Lambda$ and $g_z:=dz^2$ is the Euclidean metric on normal fibers of
$N\Lambda$. We deal with the operators
\[
    \hat L_\ep:=-(\ep^2\Delta_{\bar g}+\lambda e^{\tilde u_\ep}) \quad \mbox{and}
    \quad \tilde L_\ep := -(\ep^2\Delta + \lambda e^{u_\ep}),
\]
to which we associate the quadratic forms
\begin{eqnarray*}
    \hat Q(v) &=& \ep^2 \int_{T_\ep}|\nabla_{\bar g} v|^2\, \operatorname{dvol}_{\bar g}
    - \lambda \int_{T_\ep}e^{u_\ep}v^2 \, \operatorname{dvol}_{\bar g}\\
    \tilde Q(v) &=& \ep^2 \int_{T_\ep}|\nabla_{g} v|^2\, \operatorname{dvol}_{g}
    - \lambda \int_{T_\ep}e^{u_\ep}v^2 \, \operatorname{dvol}_{g}\\
\end{eqnarray*}
respectively, where $g$ stands for the Euclidean metric.

We know explicit eigenvalues of $\hat L_\ep$, which, together with Weyl's
approximation formula, will give us an estimate of the Morse index of
$\tilde L_\ep$.

\begin{lemma}
    There exists a constant $C>0$ such that, for all sufficiently small $\ep>0$,
    \[
        \operatorname{Index}(\tilde L_\ep)\leq \frac{C}{\ep}.
    \]
\end{lemma}

\begin{proof} Let $w\in H_0^1(T_\ep)$ with $||w||_{L^2(T_\ep)}=1$ and
    satisfying $\tilde Q(w)\leq 0$. We will give an estimate for $\hat Q(w)$.
    First,
    \begin{eqnarray}
        \label{eq-nabla-w}
        ||\nabla_g w|^2-|\nabla_{\bar g} w|^2|&\leq& C\ep |\nabla_g w|^2,\\
        |\sqrt{\det g} - \sqrt{\det \bar g}|&\leq& C\ep\sqrt{\det \bar g}
    \end{eqnarray}
    follow at once from the fact that, if one considers a parameterisation $\Phi$
    from a neighbourhood of $(y,0)\in \Lambda\times \RR^{n-1}$ into a
    neighbourhood of $y\in\RR^n$ by
    \[
        \Phi(y,z_1,\dots,z_{n-1}):=y + \sum_{i=1}^{n-1}z_i e^i(y),
    \]
    where $\{e^1,\dots, e^{n-1}\}\subset N\Lambda$ is a moving orthonormal
    frame in which each $e^j$ is a smooth section of the normal bundle of
    $N\Lambda$, one can show that the pull-back of $g$ by $\Phi$ is close
    to $\bar g$. For completeness, let us give a quantitative version of this
    statement, which is proved in~\cite{Pacard2014} (Lemma 3.1). {\color{red}
    Expresar referencia como lema.} Using~\ref{eq-nabla-w}, we have
    \begin{eqnarray*}
        \hat Q(w) &\leq & \hat Q(w)-\tilde Q(w)\\
                  &\leq & C\ep\bigg(\int_{T_\ep} \ep^2 \cdot |\, |\nabla_g
                      w|^2-|\nabla_{\bar g} w|^2|\, d\mbox{vol}_{\bar g}
                      + \int_{T_\ep} \lambda e^{u_\ep}w^2\,
                  d\operatorname{vol}_{\bar g}\bigg)\\
                  &\leq & C\ep\int_{T_\ep}(\ep^2|\nabla_g w|^2+\lambda
                  e^{u_\ep}w^2) \, d\mbox{vol}_g.
    \end{eqnarray*}
    Also, $\tilde Q(w)\leq 0$ implies
    \begin{equation}
        \int_{T_{\ep}} \ep^2|\nabla_g w|^2\, d\mbox{vol}_g \leq \lambda
        \int_{T_\ep}e^{u_\ep}w^2\, d\mbox{vol}_g \leq C'\int_{T_\ep}w^2\,
        d\mbox{vol}_{\bar g}=C',
    \end{equation}
    from where we conclude $\hat Q(w)\leq C\ep$ for a posivite constant $C$.
    This means that the index of $\tilde L_\ep$ is less than the number of
    independent eigenfunctions of $\hat L_\ep$ ssociated to eigenvalues that
    do not exceed $C\ep$. In other words,
    \begin{eqnarray*}
        \mbox{Index}(\hat L_\ep) &\leq & \#\{j\in\NN:\, \mu_i+\ep^2\nu_j\leq
        C\ep\}\\
                                 & = & \#\left\{j\in\NN:\,\nu_j\leq
                                 \frac{C}{\ep}-\frac{\mu_i}{\ep^2}\right\}\\
                                 & \leq & \#\left\{j\in\NN:\,\nu_j\leq
                                 \frac{C}{\ep}-\frac{\mu_1}{\ep^2}\right\}\\
                                 & \leq & \#\left\{j\in\NN:\,\nu_j\leq
                                     \frac{\alpha}{\ep}
                                 \right\}\approx \frac{1}{\ep},
        \end{eqnarray*}
        where the last estimate uses the fact that, from Weyl's formula, the number
        of eigenvalues of $-|\Delta_{\mathring g}|$ counted with multiplicity which
        are less than $\theta>0$ is equivalent to $\sqrt{\theta}$ as $\theta\to\infty$.

    \end{proof}

    
\subsection{Exploiting Pacard-Pacella-Sciunzi's results}
\label{sec:exploiting-pacard}

In section 5 of \cite{Pacard2014}, authors prove that any operator of the form
$-(\Delta+V(x))$ is invertible in the tube, starting from the invertibility of
$-(\frac{1}{\ep^2}\Delta_{g_z}+\Delta_{\mathring g}+V(x))$ in the same domain.
Their analysis for the power non-linearity can be applied by analogy with no
further complications to obtain the invertibility of $\tilde L_\ep$.

Let us describe the decomposition of eigenfunctions of $\tilde L_\ep$. Let
$\Phi_1:B^{n-1}\to \RR$ denote the first eigenfunction of $-(\Delta+\lambda
e^{\tilde U})$ in the ball $B^{n-1}$ with Dirichlet boundary conditions. Using
separation of variables we have that the eigenfunction of $\hat L_\ep$
associated to the eigenvalue $\mu_1+\ep^2\nu_j$ can be decomposed as
$\phi_1(z)\psi_j(y)$, where $\psi_j$ is the $j$-th eigenfunction of
$-\ep^2\Delta_{\mathring g}$ in $\Lambda$ and $\phi_1(z):=\Phi_1(z/|\ep|)$. Here
we prove a similar property for the operator $\tilde L_\ep$. We insist on the
fact that the ideas and lemmas here are a direct adaptation of the power
non-linearity treated in \ref{sec:exploiting-pacard}. First, for fixed $\ep$
define the smooth function $a$ and the operator $\tilde H_\ep$ acting on
functions in the tube by
\begin{equation}
    \operatorname{dvol}_g = \operatorname{dvol}_{\bar g}\quad \mbox{ and }
    \quad \tilde H_\ep:= a\tilde L_\ep,
\end{equation}
so that $\tilde L_\ep$ is self-adjoint with respect to $L^2(T_\ep(\Lambda),g)$
and $\tilde H_\ep$ is self-adjoint with respect to $L^2(T_\ep(\Lambda),\bar
g)$. We then have the following.

\begin{lemma}
    Let $v$ be an eigenfunction of $\tilde H_\ep$ associated to the eigenvalue
    $\gamma$, and for a function $\psi$ defined on $\Lambda$ write
    \begin{equation}
        v(y,z):=\phi_1(z)\psi(y)+w(y,z)
    \end{equation}
    with the orthogonality condition
    \begin{equation}
        \forall h \in L^2(\Lambda),\; \int_{T_\ep}w\cdot \phi_1\cdot h\, \operatorname{dvol}_{\bar g} = 0.
    \end{equation}
    Then there exists a constant $C$ such that 
    \begin{equation}
        \int_{T_\ep}(|\nabla_{\bar g}w|^2+w^2)\operatorname{dvol}_{\bar g} \leq
        C\gamma\ep\int_{T_\ep} v^2\operatorname{dvol}_{\bar g}.
    \end{equation}
\end{lemma}
We omit the proof as it follows by analogy from the proof of Lemma 5.2 in
\cite{Pacard2014}. Remark, however, that this lemma is more accurate than its
analogue, because we do not need assumptions on the eigenvalue $\gamma$ and
because the estimation is given in term of a positive power of $\ep$. In
particular, we can see the function $w$ as a small perturbation of $v$ in the
$L^2(T_\ep(\Lambda))$ sense.
\medskip

Pacard \etal also give an estimate for the change rate of the eigenvalues as
$\ep$ increases, which we adapt here as well:

\begin{lemma}
    \label{lem-partial-nu}
    If $\nu$ is an eigenvalue of $\tilde H_\ep$ such that $\nu<\alpha$, then
    there is a constant $C$ such that
    $$\frac{\partial \nu}{\partial \ep} \geq C\alpha\ep.$$
\end{lemma}
\begin{proof}
    {\color{red} To do. Paragraph of page 14 is suspicious.}
\end{proof}

    \subsection{Invertibility of $\tilde L_\ep$}

One may apply the preceding analysis to conclude the following.

\begin{theorem}
Given $N\geq 3$, there exists a set $A^N\subset (0,+\infty)$ and $N_0\in \NN$
such that, for all $\ep\in A^N$, the operator $\tilde L_\ep$ is invertible and
the norm of its inverse defined from $C^0(T_\ep(\Lambda))$ to
$C_0^1(T_\ep(\Lambda))$ (the subspace of $C^1(T_\ep(\Lambda))$ spanned by
functions that vanish on $\partial T_\ep(\Lambda)$) is bounded by a constant
times $\ep^{-1-N-N_0}$. Moreover, for $\alpha>N-1$, 
$$
\lim_{\ep\to 0}\frac{\ep -\operatorname{meas}(A^N\cap(0,\ep))}{\ep^\alpha} = 0.
$$
\end{theorem}

\begin{proof}
 Fix $\ep>0$, denote by $\Sigma_\ep$ the spectrum of $\tilde H_\ep = a\tilde
 L_\ep$ and the set of ``resonant'' values of $\ep$ by
 $$
 R_\ep:=\{\ep>0:\, 0\in S_\ep\}
 $$
 {\color{red} Shouldnt it be $\Sigma$ instead of $S$?}

 It is easy to show that if $\ep\neq R_\ep$, the norm of the inverse of $\tilde
 H_\ep$ can be estimated by a constant times $1/\delta_\ep$ where
 $\delta:=\min\{|\mu|:\, \mu\in S_\ep\}$ {\color{red} same} is the distance
 from 0 to the spectrum.  Fix now $N\geq 2$ and define, for all $\ep\in(0,1)$,
 the subset of ${}^c(R_\ep)$
 \begin{equation}
    A_\ep^N:=\{\alpha\in(\ep,2\ep):\, (\alpha-\ep^N,\alpha + \ep^N)\cap
    R_\ep=\emptyset\},
\end{equation}
which consists of values that are far enough from the resonant ones. The fact
that the Morse index of $\tilde L_\ep$ is bounded by $C/\ep$ implies that
$\ep-\mbox{meas}(A_\ep^N)$ (the quantity of negative eigenvalues of $H_\ep$ up
to a multiplicative constant) cannot belarger than a constant times
$\ep^{N-1}$. The result of Lemma \ref{lem-partial-nu} also implies that the
norm of the inverse of $\tilde H_\ep$ is bounded by a constant times
$\ep^{-1-N}$ and, as $\tilde L_\ep = a\tilde H_\ep$ where $a$ is bounded away
from 0, we have the same property for $L_\ep$. If we let
\begin{equation}
    A^N:=\bigcup_{\ep\in(0,1)} A_\ep^N,
\end{equation}
then for every $\ep\in A^n$, $\tilde L_\ep$ is invertible, when defined from
$L^2(T_\ep)$ to itself, and the norm of its inverse is bounded by a constant
times $\ep^{-1-N}$. The norm of the inverse of $\tilde{L}_\ep$ when defined
from $C^0(T_\ep)$ into $C_0^1(T_\ep))$ follows from Schauder's estimates:
indeed, up to some powers of $\ep$, say $N_0$, we have the same property. This
completes the proof.
\end{proof}



    We have therefore proven that, in dimensions 2 and 3, there exist two
    solutions to the problem given that $\ep$ is away from the set of values
    that lead to resonance phenomena: the stable one discussed
    in~\ref{sec:stable-sol-in-tube} and the unstable one, with Morse index
    $\approx 1/\ep$.
