% !TeX root = ../main.tex

\subsection{The stable branch}
\label{sec:stable-branch}

Let $L(u):=-\Delta u - \lambda e ^ u$. We say that a solution $u$ of
$(G_\lambda)$ is \textit{stable} if the linearized operator of $L$ at the point
$u$ is positive definite. This notion of stability comes from the following
argument: For $\Omega$ consider the functional $\EE:C_0^2(\Omega)\to\RR$
defined by
\begin{equation}
    \EE(u):= \frac{1}{2}\int_\Omega |\nabla u|^2\, dx - \int_\Omega e^u\, dx.
\end{equation}
We say that $u$ is a critical point of $\EE$ if for every $\phi\in
C_0^2(\Omega)$, $0$ is a critical point of $E:\RR\to \RR$ given by
$E(t):=\EE(u+t\phi)$. Indeed, the equation $E'(0)=0$ gives
\begin{equation}
    \int_\Omega (-\Delta u-e^u)\,\phi \, dx = 0\; \mbox{ and }\;
    E''(0)=\int_\Omega|\nabla u|^2 - \int_\Omega e^u\phi^2.
\end{equation}

It is hence natural to define stability as follows.

\begin{definition} Let $\Omega$ be an open set of $\RR^n$, and $u\in
    C_0^2(\Omega)$ a solution of $-\Delta u = f(u)$. We say that $u$ is stable
    if
    $$
    Q_u(\phi):=\int_\Omega|\nabla \phi|^2\;dx - \int_\Omega f'(u)\,\phi^2\;dx
    \geq 0
    $$
    for all $\phi\in C_c^1(\Omega)$ (or $\phi\in H_0^1(\Omega)$ if $\Omega$ is
    bounded).
\end{definition}

For the sake of completeness, let us give some properties of stable solutions.

\begin{proposition} Local minimisers of the energy $\EE$ are stable.
\end{proposition}

\begin{proposition} A $C_0^2(\Omega)$ solution of $-\Delta u = f(u)$ is stable
    if and only if $\lambda_1(-\Delta u-f'(u),\omega)\geq 0$ for every bounded
    subdomain $w$ of $\Omega$ (or simply, $\omega=\Omega$ if $\Omega$ is
    bounded).
\end{proposition}
\begin{proposition} A $C_0^2(\Omega)$ solution of $-\Delta u = f(u)$ is stable
    if and only if it exists $v\in C^2(\Omega)$, $v>0$, and $-\Delta v -f'(u)\,
    v\geq 0.$
\end{proposition}
\begin{proposition} There is one unique stable solution of $(G_\lambda)$ for
    every admissible $\lambda$, which is a minimiser of the energy and the
    solution with the smallest $L^\infty(\Omega)$ norm.
\end{proposition}

For more properties and proofs we recomend \cite{stable-solutions-elliptic},
$\S 1$. We define also the lower (or stable) branch of the curve
$$\mathcal S:=\{(\lambda,||u_\lambda||_{L^\infty(\Omega)}),\, u_\lambda\mbox{ is a
    solution of }(G_\lambda)\}.
$$
It turns out (see \cite{stable-solutions-elliptic}, \S 3.3), that for every $n$
there exists a maximum value $\lambda_\ast$ such that $\mathcal{S}$
is a smooth curve connecting $(0,0)$ with $(\lambda_\ast,
||u_{\lambda_\ast}||_{L^\infty(\Omega)})$, which is unbounded in the $||\cdot
||_{L^\infty}$ direction if $n\geq 10$.

