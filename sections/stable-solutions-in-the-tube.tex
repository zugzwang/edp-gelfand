\section{The stable solution in the tube}
\label{sec:stable-sol-in-tube}

In this section, we establish the following:

\begin{theorem}
\label{thm:unique-stable-sol}
For small $\ep$, there exists a unique stable solution to
\begin{equation}
\label{eq:small-gelfand}
\left\{
\begin{array}{cc}
\Delta u + \frac{\lambda}{\ep^2}e^u=0 & \mbox{ in }T_\ep(\Lambda),\\
u = 0 & \mbox{ on }\partial T_\ep(\Lambda).
\end{array}
\right.
\end{equation}
\end{theorem}

\noindent An outline of the proof follows:

\begin{itemize}
\item Let $U$ be the unique radial stable solution of $(G_\lambda)$ for the
$(n-1)$-dimensional ball. We choose $u_\ep$, a rescaled version of $U$ that
travels along $\Lambda$, this is, for each section of the normal
bundle of $\Lambda$, consider the $(n-1)$-dimensional ball of radius $\ep$
centered about $\Lambda$ and use a copy of $U$.
\item The function $u_\ep$ verifies approximately the equation and it is stable
in the sense of section \ref{sec:stable-branch}. To prove this, we find a
super-solution to the problem and use a version of the maximum principle.
\item We use a fixed-point argument to prove the existence of a genuine solution
to \ref{eq:small-gelfand} of the form $u_\ep+v$, where $v$ is small. The
fixed-point theorem ensures the uniqueness of the perturbation if $v$ is small
enough. Finally, we show that larger perturbations do not lead to stable
solutions, proving the result.
\end{itemize}

Let us first collect some lemmas that will help us prove
\ref{thm:unique-stable-sol}.
\subsection{Some elliptic estimates}
Throughout this section, $||\cdot||$
stands for the $L^\infty$ norm in the tube
$T_\ep$, unless stated otherwise. Let $U(r)$ be the unique (radial) stable
solution of $(G_\lambda)$ for $\Omega
= B^{n-1}$, \ie
\begin{equation}
\left\{\begin{array}{cc}
\Delta U + \lambda\, e^U=0 & \mbox{ in }B^{n-1},\\
U = 0 & \mbox{ on }\partial B^{n-1},
\end{array}\right.
\end{equation}

\noindent and define $u_\ep:T_\ep\to \RR$ by
$$
u_\ep(y,z):=U\bigg(\frac{\mbox{dist}(z,\Lambda)}{\ep}\bigg).
$$
As the non-linearity $\lambda\,e^u$ is not multiplicative, we have that the
function $u_\ep$ will not verify the Gelfand equation in the tube, nor an
approximation, but the following estimate.

\begin{lemma}\label{lem-bound_Cepsilon} There is a constant $C$ such that
\begin{equation}
|\ep^2\Delta u_\ep+\lambda\, e^{u_\ep}|\leq C\ep.
\end{equation}
\end{lemma}

\begin{proof} It follows from the definition of $u_\ep$ that
$\ep^2\Delta_{g_z}u_\ep + \lambda\, e^{u_\ep}=0$, and note that $u_\ep$ does not
depend on the parameter $t$. Using the expression of the Euclidean Laplacian in
Fermi coordinates, we have
$$\ep^2\, \Delta u_\ep+\lambda e^{u_\ep} = (\ep^2\, \Delta_{g_z} u_\ep +
\lambda e^{u_\ep})+\ep^2(\partial_{tt}+D)u_\ep = \ep^2 D\, u_\ep,
$$
therefore, the estimate
\begin{eqnarray}
|\ep^2\Delta u_\ep + \lambda e^{u_\ep}|&\leq& \ep^2\, ||Du_\ep|| \\
&\leq& \sum_{i=1}^{n-1} \big(\ep^2||z_iD^{(2)}u_\ep||+\ep^2||D^{(1)}u_\ep||\big)\\
    &=& \sum_{i=1}^{n-1}\big(\ep
    ||x_iD^{(2)}U||_{L^\infty(B)}+\ep||D^{(1)}U||_{L^\infty(B)}\big)
\end{eqnarray}
holds and the result follows.
\end{proof}
\medskip

The stability of $U$ implies that there exist $\mu_1>0$ and $\phi_1>0$ such that
\begin{equation}
\left\{\begin{array}{cc}
-(\Delta+\lambda e^U)\phi_1=\mu_1\phi_1 & \mbox{ in }B^{n-1},\\
\phi=0 & \mbox{ on }\partial B^{n-1},
\end{array}\right.
\end{equation}
and hence the linearized operator about $U$ is invertible. This proves the
existence of a function $W$, which is a super-solution to the linearized
equation in $B^{n-1}$, verifying
\begin{equation}
\left\{\begin{array}{cc}
-(\Delta+\lambda e^U)W=1 & \mbox{ in }B^{n-1},\\
W=0 & \mbox{ on }\partial B^{n-1}.
\end{array}\right.
\end{equation}

\begin{lemma} There is a constant $B$ such that $|W|\leq B, |\nabla W|\leq
B, |\Delta W|\leq B$.
\end{lemma}
\begin{proof}
By the GNN theorem, $W$ is radially symmetric, positive and $\partial_r W<0$,
which gives $|W|\leq W(0)$. Write $W(z)=a(r)$, then $a$ solves
\begin{equation}
\left\{
\begin{array}{lc}
a''(r)+\frac{n-1}{r^2}a'(r)+\lambda e^Ua+1=0, & 0<r<1,\\
a(1) = 0, a'(0) = 0. &
\end{array}
\right.
\end{equation}
As $a(r)$ is bounded and positive, we have 
\begin{equation}
    \label{eq:bounding-a'}
-C\leq a''(r)+\frac{n-1}{r^2}a'(r)\leq -1
\end{equation}
for a positive constant $C$.
Multiplying \ref{eq:bounding-a'} by $e^{-\frac{n-1}{r}}$ we have $
-Ce^{-\frac{n-1}{r}}\leq \big(e^{-\frac{n-1}{r}}a'(r)\big)'\leq
e^{-\frac{n-1}{r}}$, \ie  
\begin{eqnarray*}
    |a'(r)|\leq  C_1\cdot e^{\frac{n-1}{r}}\int_0^re^{-\frac{n-1}{t}}\, dt \leq
    C_1  
\end{eqnarray*}
for a constant $C_1=\max(C,1)$. This yields $|\nabla W(z)|=|a(r)|\leq C_1$.
Finally, write $|\Delta W|\leq 1+e^UW\leq C_2$ for a positive constant $C_2$
and define $B:=\max(C_1,C_2,W(0))$.

\end{proof}

Similarly, define $w_\ep:T_\ep\to \RR$ by
$w_\ep(y,z):=W\big(\frac{\operatorname{dist}(z,\Lambda)}{\ep}\big)$. The
corresponding estimate for $w_\ep$ will be given by the following. 

\begin{lemma}
For sufficiently small $\ep$, the function $w_\ep$ verifies
\begin{equation}
    \label{eq-lemma--1/2}
\ep^2 \Delta w_\ep + \lambda e^{u_\ep}w_\ep \leq -1/2.
\end{equation}
\end{lemma}

\begin{proof}
Again note that $w_\ep$ does not depend on the parameter $t$ and that
$-(\ep^2\Delta_{g_z}w_\ep + e^{u_\ep}w_\ep)=1$. We then have that $(\ep^2\Delta
+ \lambda e^{u_\ep})w_\ep = (\ep^2\Delta_{g_z}+\lambda
e^{u_\ep})w_\ep+\ep^2(\partial_{tt} + D)w_\ep = -1+\ep^2Dw_\ep$. As $|\ep^2
Dw_\ep|\leq \ep C'$ for a constant $C'$ (as in the proof of lemma 1)
, the result follows for small $\ep$.
\end{proof}

Consider now the invertibility problem for the linearized operator
$L_\ep:=-(\ep^2 \Delta + e^{u_\ep})$ in $T_\ep(\Lambda)$: Given a smooth $f$ in
the tube, find $\phi$ such that
\begin{equation}
    \label{eq-lin-ep}
    \left\{
    \begin{array}{cc}
        L_\ep\phi = f & \mbox{ in }T_\ep(\Lambda),\\
        \phi = 0& \mbox{ on } \partial T_\ep(\Lambda).
    \end{array}
    \right.
\end{equation}

\begin{lemma} Let $\phi$ solve \ref{eq-lin-ep}, then
    \label{lem-phi}
    $\phi\leq 2 ||f||_{L^\infty(T_\ep)}\cdot w_\ep. $
    Moreover, $L_\ep$ is invertible given that $\ep$ is small enough.
\end{lemma}

\begin{proof} Let us first give the a-priori estimate by means of the maximum
    principle. From equations \ref{eq-lemma--1/2} and \ref{eq-lin-ep} we have
    \begin{equation}
        L_\ep w_\ep \geq \frac{1}{2}\;\mbox{ and }\;
        L_\ep\bigg(\frac{\phi}{2||f||}\bigg)\leq \frac{1}{2}.
    \end{equation}
    By addition, we have $L_\ep\big(w_\ep-\frac{1}{2||f||}\phi\big)\geq 0$.
    Recall the strong maximum principle: If $u$ is a smooth function verifying
    $-\Delta u\geq 0$ in a connected domain $\Omega$ of $\RR^n$, then if $u$
    attains a minimum in $\Omega$, $u$ is constant. The same conclusion holds
    if $-\Delta$ is replaced by $-\Delta+a(x)$ with $a\in L^p(\Omega)$ and
    $p\leq n/2$ (see \cite{1980JPhA...13..417H} for a proof of this theorem with
    applications to a Helium-like system). Note that the conclusion holds
    regardless of the sign of $a(x)$. Apply the strong maximum principle to the
    operator $L_\ep$ to conclude that either $w_\ep-\frac{1}{2||f||}\phi$ is
    constant or it attains a minimum in $\partial T_\ep(\Lambda)$. Because both
    $w_\ep$ and $\phi$ vanish on the boundary, in both cases
    $w_\ep-\frac{1}{2||f||}\phi\geq 0$, proving the first assertion.
    To prove that $L_\ep$ is invertible, write the
    problem in Fermi coordinates $(y,z)\in\Lambda\times B^{n-1}$:
    \begin{equation}
        \left\{
            \begin{array}{cc}
                -(\ep^2 \Delta_{\bar g}+\lambda e^{u_\ep})\phi - \ep^2\, D=f & \mbox{ in }T_\ep(\Lambda),\\
                \phi=0 & \mbox{ on }\partial T_\ep(\Lambda).
            \end{array}
            \right.
    \end{equation}

    Recall that $D$ is of the form $\sum_{i=1}^{n-1} z_iD^{(2)}+D^{(1)}$ where
    $D^{(i)}$ is an $i$--differential operator with smoothly bounded
    coefficients. In order to work in a domain with no dependence in $\ep$, use the
    scaling $z\mapsto z/\ep$ and define $\Phi(y,z):=\phi(y,\ep z),
    F(y,z):=f(y,\ep z)$.
    After simple manipulation, the problem reads
    \begin{equation}
        \left\{
            \begin{array}{cc}
                (L-\ep D)\Phi = F& \mbox{ in }T_1(\Lambda),\\
                \Phi=0 & \mbox{ on }\partial T_1(\Lambda).
            \end{array}
            \right.
    \end{equation}
    where $L=-(\Delta_{\bar g} + \lambda e^U)$ is invertible by hypothesis.
    Indeed, its eigenvalues are numbers of the form $\mu_i+\ep^2 \nu_j$ where
    $0<\mu_1\leq \mu_2\leq \mu_3\leq \dots$ are the eigenvalues of
    $-(\Delta_{g_z}+\lambda e^U)$ in $B^{n-1}$ and $0=\nu_1<\nu_2\leq \nu_3\leq
    \dots$ are the eigenvalues of $-\Delta_{\mathring g}$ on $\Lambda$.
    Therefore, $L-\ep D$ is a small perturbation of $L$, and we conclude that
    it is invertible for small $\ep$ after a simple fixed-point argument:
    Write $\Phi+L^{-1}F+\Psi$, then $\Psi$ solves 
    \begin{equation}
        \left\{
            \begin{array}{cc}
                \Psi = \ep L^{-1}D(L^{-1}F+\Psi)& \mbox{ in }T_1(\Lambda),\\
                \Psi=0 & \mbox{ on }\partial T_1(\Lambda).
            \end{array}
            \right.
    \end{equation}
    Using the fact that the norm of the inverse of $L$ is controlled by
    $1/\mu_1$, it is straightforward to show that the right-hand operator (to
    which we associate the $||\cdot||_\infty$ norm in the tube) is a
    contraction mapping for small $\ep$ that maps the space of bounded
    functions in the tube into itself. Thus $\Psi$ exists, allowing to conclude.
\end{proof}


\subsection{Proof of theorem \ref{thm:unique-stable-sol}}

\textit{Proof:} Write a perturbation of the solution $u=u_\ep+v$. Problem
    \ref{eq:small-gelfand} reduces to finding $v$ such that
    \begin{equation}
        \label{eq-fixed-point-H}
        \left\{
            \begin{array}{cc}
                \Delta (u_\ep+v)+\frac{\lambda}{\ep^2}e^{u_\ep+v}=0& \mbox{ in }T_\ep(\Lambda),\\
                v=0 & \mbox{ on }\partial T_\ep(\Lambda).
            \end{array}
            \right.
    \end{equation}
    Rewrite the differential equation as $(\ep^2\Delta + \lambda
    e^{u_\ep})v+(\ep^2\Delta u_\ep + \lambda e^{u_\ep})+\lambda
    e^{u_\ep}(e^v-1-v)=0$, and, recalling that $L_\ep$ is invertible for
    small $\ep$, define the following operator
    \begin{equation}
        H_\ep:= v\mapsto L_\ep^{-1}((\ep^2\Delta u_\ep + \lambda
        e^{u_\ep})+\lambda e^{u_\ep}(e^v-1-v)).
    \end{equation}
    Thus, any solution of \ref{eq-fixed-point-H}  is a fixed point of $H_\ep$. Let us introduce the space of functions
    $$\AAA_\ep :=\{v\in L^\infty(T_\ep(\Lambda)),\exists\, C \in \RR, ||v||\leq
    C\ep\},$$
    to which we associate the $L^\infty$ norm in the tube.
    % {\color{red}The space is not well defined because of the $\exists C$.}
    The following lemmas show that $H_\ep$ is a contraction mapping in
    $\AAA_\ep$. First, the fact that $H_\ep(\AAA_\ep)\subset \AAA_\ep$ for
    sufficiently small $\ep$ is a direct consequence of Lemma \ref{lem-phi} for
    a function in $\AAA_\ep$. 
    \begin{lemma}
        Let $v$ verify \ref{eq-fixed-point-H} and define $B=\max(1,2||\lambda
        e^U W||_{L^\infty(B^{n-1})})$. If $|v|\leq 1/B$, then $|v|\leq C\ep
        w_\ep$, where $C$ is a constant close to the constant from Lemma
        \ref{lem-bound_Cepsilon}.
    \end{lemma}
    \begin{proof} Lemma \ref{lem-phi}, applied for $v$ and $f=(\ep^2\Delta
        u_\ep+\lambda e^{u_\ep})+ \lambda e^{u_\ep}(e^v-1-v)$ gives
    \begin{eqnarray}
        v &\leq & 2 ||(\ep^2\Delta u_\ep + \lambda e^{u_\ep})+\lambda
        e^{u_\ep}(e^v-1-v)||w_\ep\\
          &\leq & C\ep w_\ep + B(e^v-1-v),
    \end{eqnarray}
    for a constant $C$ from Lemma \ref{lem-bound_Cepsilon} not depending on
    $\ep$. Using the fact that $e^t-1-t<t^2$ for $t\in[-1,1]$ (the first
    non-zero root of $g(t):=t^2-(e^t-1-t)$ is greater than $\ln(3)$), we
    conclude that, as $|v|\leq 1$, $|v|\leq C\ep w_\ep+Bv^2$. Thus,
    \begin{equation}
        \big(|v|-C\ep w_\ep+O(\ep^2)\big)\cdot\big( |v| -\frac{1}{B}-C\ep
        w_\ep+O(\ep^2)\big)\geq 0,
    \end{equation}
    from where the result follows. 
\end{proof}

\begin{lemma}
    $H_\ep$ is a contraction mapping of $\AAA_\ep$.
\end{lemma}
\begin{proof}
    Take $f,g\in \AAA_\ep$, then,
    $$||H_\ep(f-g)||=||L_\ep^{-1}\big(\lambda e^{u_\ep}(e^f-e^g -
    (f-g))\big)||.  $$
    The convexity of the exponential function implies that for any reals
    $\alpha,\beta$ with $\alpha>\beta$ we have $e^{\alpha}-e^{\beta}\leq
    e^{\alpha}(\alpha-\beta)$, and, taking $\beta=0$, $e^\alpha-1\leq \alpha e^\alpha$, therefore,
    \begin{eqnarray*}
        ||H_\ep(f-g)||&\leq& C\lambda ||e^{u_\ep}||\cdot ||e^{\max(f,g)}-1||\cdot
    ||f-g||\\
                      &\leq& C'\ep||f-g||,
    \end{eqnarray*}
    for a constant $C'$ not depending on $\ep$.
\end{proof}
This proves that there exists a
    unique genuine solution of $\ref{eq:small-gelfand}$ of the form $u_\ep+v$
    with $v$ small. The stability of this solution comes from the fact that the
    spectrum of the operators $-(\Delta+\lambda e^U)$ in $T_1(\Omega)$ and
    $-(\Delta + \lambda e^{u_\ep}$ in $T_\ep(\Omega)$ are equal: We have
    $$-(\Delta + \lambda e^{u_\ep+v})=-(\Delta + \lambda e^{u_\ep})+O(\ep)
    \mbox{ in } T_\ep(\Omega),$$
    and therefore the eigenvalues of the left-hand operator are close to the
    sequence $\mu_i$, which does not depend on $\ep$. In particular, they form
    a sequence of positive values for small $\ep$, which implies precisely that
    the associated quadratic form is positive definite. This allows to conclude
    that, for sufficiently small $\ep$, there exists a unique small perturbation
    that leads to a genuine stable solution of \ref{eq:small-gelfand}. This
    uniqueness holds only in a neighbourhood of $u_\ep$; it remains the
    question of whether there are stable solutions far from $u_\ep$. We answer
    negatively with the following argument: Suppose that there are two distinct
    stable solutions of \ref{eq:small-gelfand}. Their difference $v:=u_2-u_1$
    verifies
    $$-\ep^2 \Delta v = \lambda(e^{u_2}-e^{u_1}).$$
    Multiplying the above equation by the positive part of $v$ and integrating
    in the tube gives
    \begin{equation}
        \label{eq:unique-stable}
        \ep^2\int_{T_\ep}|\nabla v_+|^2\,dx = \lambda
    \int_{T_\ep}(e^{u_2}-e^{u_1})\cdot v_+\, dx.
\end{equation}
    As $u_2$ is stable, we have $\ep^2\int_{T_\ep}|\nabla v_+|^2\geq \lambda
    \int_{T_\ep} e^{u_2}\cdot v_+^2\, dx$. Plugging this inequality into
    \ref{eq:unique-stable} yields
    $$
    0 \leq \lambda \int_{T_\ep}(e^{u_2}-e^{u_1}-e^{u_2}v_+)v_+\, dx.
    $$
    Note that the integrand is a negative number by strict convexity of the
    exponential function, therefore, $v_+=0$. Changing $u_1$ and $u_2$ gives
    $v_-=0$, completing the proof of Theorem \ref{thm:unique-stable-sol}.
    \hfill $\blacksquare$
