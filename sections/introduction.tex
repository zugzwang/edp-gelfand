% !TeX root = ../main.tex

\section{Introduction}
Let $\Omega$ be a domain of $\RR^n$, and for $\lambda>0$ consider the Gelfand 
problem
\begin{equation}
    \label{eq-g-lambda}
(G_\lambda)\;\left\{ 
    \begin{array}{cl} \Delta u + \lambda e^u=0 & \mbox{ in }\Omega,\\
        u=0 & \mbox{ on }\partial \Omega.
    \end{array}\right.
\end{equation}
Also, let $\Lambda$ be a smooth closed curve embedded in $\RR^n$, and given
$\ep>0$ we define the tubular neighbourhood of radius $\ep$ centered about
$\Lambda$ by
\begin{equation}
    \label{eq-tube}
    T_\ep(\Lambda):= \{x\in \RR^n: \dist(x,\Lambda)<\ep\}.
\end{equation}

In this article, we study the Gelfand problem $(G_\lambda)$ in this kind of
tubular domains. Let us first give some properties of this problem
in all generality, assuming that $\Omega$ is
bounded and connected.

\begin{itemize}
    \item[(i)] If $u$ solves $(G_\lambda)$, $\lambda>0$ if and only if $u>0$.
        This is a straightforward consequence of the maximum principle.
    \item[(ii)] If $(G_\lambda)$ admits a solution $u$, then
        $\lambda \leq \lambda_1(\Omega)$, where $\lambda_1(\Omega)$
        stands for the first eigenvalue of the Dirichlet Laplacian in $\Omega$.
        Indeed, let $\phi>0$ be a principal eigenfunction of the Laplacian in
        $\Omega$ with Dirichlet boundary conditions, and multiply $(G_\lambda)$ by
        $\phi$. Integrating by parts and using $u<e^u$ gives
        $(-\lambda_1(\Omega)+\lambda)\int_\Omega \phi\,u < 0.$
    \item[(iii)] With implicit function methods, one can establish a local
        solution curve $(\lambda, u)\in [0,\infty)\times C(\bar\Omega),$ which
        emanates from the stable solution $\lambda=0,u=0$. Because of (ii) this
        curve is contained in $[0,\lambda_1(\Omega)]\times C(\bar\Omega)$.
    % \item[(iv)] Fix $\lambda$ such that $(G_\lambda)$ admits non-trivial
    %     solutions and let $u_\lambda$ verify $(G_\lambda)$. The mapping
    %     $u_\lambda\mapsto ||u_\lambda||_{L^\infty}$ is injective. {\color{red}
    %     why?}
\end{itemize}

We can therefore identify $(\lambda,u)\in [0,\lambda_1]\times C^2(\bar\Omega)$
with a pair $(\lambda, ||u||_{L^\infty})\in[0,\lambda_1]\times \RR^+$, and draw
the bifurcation diagram in the plane. However, determining multiplicity of these
solutions require an understanding in the general domain $\Omega$. Nevertheless,
theorem 4.6 of \cite{isoperimetric-inequalities} implies that the spectrum of
$(G_\lambda)$ (\ie the values for which $(G_\lambda)$ is solvable) is an
interval $[0,\lambda_\ast(\Omega))$, the value $\lambda_\ast$ depending heavily
on the geometry of $\Omega$.
