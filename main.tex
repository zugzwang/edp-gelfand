
\documentclass[leqno]{article}


% !TEX root = ../main.tex

%%%%%%%%%%%%%%%%%%%%%%%%%%%%%%%%
%			PACKAGES
%%%%%%%%%%%%%%%%%%%%%%%%%%%%%%%%

\usepackage{amsmath}
\usepackage{amsthm}
\usepackage{amssymb}
% \usepackage{tikz}
% \usetikzlibrary{shapes.geometric, arrows, calc, positioning, shapes.misc}
\usepackage{amsfonts}
\usepackage{hyperref}
\usepackage{multirow}
\usepackage{booktabs}
% \usepackage{algorithm}

% For inline if x then y
% \usepackage[noend]{algpseudocode}

% For colored code
\usepackage{color}
\usepackage{listings}

% For margin commenting. Erase this before final submission.
\usepackage{marginnote}

\usepackage{pgfplots}
 % !TEX root = ../main.tex



\author{Francisco Jos\'e Vial Prado \thanks{DCC,  Pontificia
Universidad Cat\'olica de Chile (\texttt{fovial@uc.cl}).}}
\date{}
% \institute{Pontificia Universidad Católica de Chile and IMFD Chile}
% \author{\small Louis Goubin\inst{1} \and
% Cyril Hugounenq\inst{2} \and
% Mart\'{i}n Ugarte\inst{3, 4} \and
% Francisco Vial-Prado\inst{3, 4}
% }
%
% \institute{\footnotesize 
% Laboratoire de Math\'{e}matiques de Versailles, UVSQ, Universit\'{e} Paris-Saclay, France 
% \and 
% Institut Fourier, Univ. Grenoble Alpes, France
% \and
% DCC, Pontificia Universidad Cat\'{o}lica de Chile, Santiago de Chile
% \and
% Instituto Milenio Fundamentos de los Datos, Chile %\url{imfd.cl}
% }
%
 % !TEX root = ../main.tex

%Equations
\numberwithin{equation}{section}
%NOTES
\newcommand{\martin}[1]{\marginnote{\color{olive}Martin: #1}}
\newcommand{\cyril}[1]{\marginnote{\color{-olive}Cyril: #1}}
\newcommand{\francisco}[1]{\marginnote{\color{orange!40!olive}Francisco: #1}}
\newcommand{\louis}[1]{\marginnote{\color{blue}Louis: #1}}

\newcommand{\fg}[1]{{\color{-orange!40!-olive}\textbf{F and G: #1}}}

\newcommand{\RR}{\mathbb{R}}      % for Real numbers
\newcommand{\ZZ}{\mathbb{Z}}      % for Integers
\newcommand{\NN}{\mathbb{N}}      % for Naturals
\newcommand{\QQ}{\mathbb{Q}}      % for Rationals
\newcommand{\ee}{\mathbf{e}}
\newcommand{\meas}{\mbox{meas}}
\newcommand{\ep}{\varepsilon}
\newcommand{\dist}{\mbox{dist}}
\newcommand{\EE}{\mathcal{E}}      % for Naturals
\newcommand{\AAA}{\mathcal{A}}      % for Naturals

\newtheorem{definition}{Definition}
\newtheorem{proposition}{Proposition}
\newtheorem{theorem}{Theorem}
\newtheorem{lemma}{Lemma}
\newcommand{\sk}{{\sf sk}}
\newcommand{\pk}{{\sf pk}}
\newcommand{\Dec}{{\sf Dec}}
\newcommand{\Enc}{{\sf Enc}}
\newcommand{\Eval}{{\sf Eval}}
\newcommand{\KeyGen}{{\sf Keygen}}
\newcommand{\BB}{{\mbox{\bf B}}}

\newcommand{\rot}{\operatorname{rot}}
\newcommand{\Samp}{{\sf Samp}}

\newcommand{\ie}{i.e.$\!$ }
\newcommand{\eg}{e.g.$\!$ }
\newcommand{\etal}{et al.$\!$ }




\begin{document}


\title{The Gelfand Problem in Tubular Domains}

\maketitle


\abstract{
    We construct stable solutions of $\Delta u + \lambda e^u=0$ with
    Dirichlet boundary conditions in small tubular domains (i.e. geodesic
    $\ep$--neighbourhoods of a curve $\Lambda$ embedded in $\RR^n$), adapting
    the arguments of Pacard-Pacella-Sciunzi. We also show unicity of these
    solutions, in particular, we show that the stable branch of the bifurcation
    diagram is similar to the well-known nose-shaped diagram of the standard
    Gelfand problem in the unit ball. In this work, $\Lambda$ can be replaced
    by any compact smooth manifold embedded in $\RR^n$.
}

    % We construct multiple solutions of $\Delta u + \lambda e^u=0$ with
    % Dirichlet boundary conditions in small tubular domains (i.e. geodesic
    % $\ep$--neighbourhoods of a curve $\Lambda$ embedded in $\RR^n$), adapting
    % arguments of Pacard-Pacella-Sciunzi.  Moreover, in dimension less than 10
    % and $\ep$ outside a resonant set accumulating around 0, we construct
    % unstable solutions, proving that the bifurcation diagram of this problem is
    % similar to the nose-shaped diagram of the standard Gelfand problem in the
    % unit ball, for every dimension. In this work, $\Lambda$ can be replaced
    % by any compact smooth manifold embedded in $\RR^n$.

\smallskip

% \textbf{Keywords:} \textit{lattice-based cryptography, fully homomorphic
% encryption, key-recovery attacks}

% !TeX root = ../main.tex

\section{Introduction}
Let $\Omega$ be a domain of $\RR^n$, and for $\lambda>0$ consider the Gelfand 
problem
\begin{equation}
    \label{eq-g-lambda}
(G_\lambda)\;\left\{ 
    \begin{array}{cl} \Delta u + \lambda e^u=0 & \mbox{ in }\Omega,\\
        u=0 & \mbox{ on }\partial \Omega.
    \end{array}\right.
\end{equation}
Also, let $\Lambda$ be a smooth closed curve embedded in $\RR^n$, and given
$\ep>0$ we define the tubular neighbourhood of radius $\ep$ centered about
$\Lambda$ by
\begin{equation}
    \label{eq-tube}
    T_\ep(\Lambda):= \{x\in \RR^n: \dist(x,\Lambda)<\ep\}.
\end{equation}

In this article, we study the Gelfand problem $(G_\lambda)$ in this kind of
tubular domains. Let us first give some properties of this problem
in all generality, assuming that $\Omega$ is
bounded and connected.

\begin{itemize}
    \item[(i)] If $u$ solves $(G_\lambda)$, $\lambda>0$ if and only if $u>0$.
        This is a straightforward consequence of the maximum principle.
    \item[(ii)] If $(G_\lambda)$ admits a solution $u$, then
        $\lambda \leq \lambda_1(\Omega)$, where $\lambda_1(\Omega)$
        stands for the first eigenvalue of the Dirichlet Laplacian in $\Omega$.
        Indeed, let $\phi>0$ be a principal eigenfunction of the Laplacian in
        $\Omega$ with Dirichlet boundary conditions, and multiply $(G_\lambda)$ by
        $\phi$. Integrating by parts and using $u<e^u$ gives
        $(-\lambda_1(\Omega)+\lambda)\int_\Omega \phi\,u < 0.$
    \item[(iii)] With implicit function methods, one can establish a local
        solution curve $(\lambda, u)\in [0,\infty)\times C(\bar\Omega),$ which
        emanates from the stable solution $\lambda=0,u=0$. Because of (ii) this
        curve is contained in $[0,\lambda_1(\Omega)]\times C(\bar\Omega)$.
    % \item[(iv)] Fix $\lambda$ such that $(G_\lambda)$ admits non-trivial
    %     solutions and let $u_\lambda$ verify $(G_\lambda)$. The mapping
    %     $u_\lambda\mapsto ||u_\lambda||_{L^\infty}$ is injective. {\color{red}
    %     why?}
\end{itemize}

We can therefore identify $(\lambda,u)\in [0,\lambda_1]\times C^2(\bar\Omega)$
with a pair $(\lambda, ||u||_{L^\infty})\in[0,\lambda_1]\times \RR^+$, and draw
the bifurcation diagram in the plane. However, determining multiplicity of these
solutions require an understanding in the general domain $\Omega$. Nevertheless,
theorem 4.6 of \cite{isoperimetric-inequalities} implies that the spectrum of
$(G_\lambda)$ (\ie the values for which $(G_\lambda)$ is solvable) is an
interval $[0,\lambda_\ast(\Omega))$, the value $\lambda_\ast$ depending heavily
on the geometry of $\Omega$.

\section{Preliminaries}
\label{sec-preliminaries}

% !TeX root = ../main.tex

\subsection{The stable branch}
\label{sec:stable-branch}

Let $L(u):=-\Delta u - \lambda e ^ u$. We say that a solution $u$ of
$(G_\lambda)$ is \textit{stable} if the linearized operator of $L$ at the point
$u$ is positive definite. This notion of stability comes from the following
argument: For $\Omega$ consider the functional $\EE:C_0^2(\Omega)\to\RR$
defined by
\begin{equation}
    \EE(u):= \frac{1}{2}\int_\Omega |\nabla u|^2\, dx - \int_\Omega e^u\, dx.
\end{equation}
We say that $u$ is a critical point of $\EE$ if for every $\phi\in
C_0^2(\Omega)$, $0$ is a critical point of $E:\RR\to \RR$ given by
$E(t):=\EE(u+t\phi)$. Indeed, the equation $E'(0)=0$ gives
\begin{equation}
    \int_\Omega (-\Delta u-e^u)\,\phi \, dx = 0\; \mbox{ and }\;
    E''(0)=\int_\Omega|\nabla u|^2 - \int_\Omega e^u\phi^2.
\end{equation}

It is hence natural to define stability as follows.

\begin{definition} Let $\Omega$ be an open set of $\RR^n$, and $u\in
    C_0^2(\Omega)$ a solution of $-\Delta u = f(u)$. We say that $u$ is stable
    if
    $$
    Q_u(\phi):=\int_\Omega|\nabla \phi|^2\;dx - \int_\Omega f'(u)\,\phi^2\;dx
    \geq 0
    $$
    for all $\phi\in C_c^1(\Omega)$ (or $\phi\in H_0^1(\Omega)$ if $\Omega$ is
    bounded).
\end{definition}

For the sake of completeness, let us give some properties of stable solutions.

\begin{proposition} Local minimisers of the energy $\EE$ are stable.
\end{proposition}

\begin{proposition} A $C_0^2(\Omega)$ solution of $-\Delta u = f(u)$ is stable
    if and only if $\lambda_1(-\Delta u-f'(u),\omega)\geq 0$ for every bounded
    subdomain $w$ of $\Omega$ (or simply, $\omega=\Omega$ if $\Omega$ is
    bounded).
\end{proposition}
\begin{proposition} A $C_0^2(\Omega)$ solution of $-\Delta u = f(u)$ is stable
    if and only if it exists $v\in C^2(\Omega)$, $v>0$, and $-\Delta v -f'(u)\,
    v\geq 0.$
\end{proposition}
\begin{proposition} There is one unique stable solution of $(G_\lambda)$ for
    every admissible $\lambda$, which is a minimiser of the energy and the
    solution with the smallest $L^\infty(\Omega)$ norm.
\end{proposition}

For more properties and proofs we recomend \cite{stable-solutions-elliptic},
$\S 1$. We define also the lower (or stable) branch of the curve
$$\mathcal S:=\{(\lambda,||u_\lambda||_{L^\infty(\Omega)}),\, u_\lambda\mbox{ is a
    solution of }(G_\lambda)\}.
$$
It turns out (see \cite{stable-solutions-elliptic}, \S 3.3), that for every $n$
there exists a maximum value $\lambda_\ast$ such that $\mathcal{S}$
is a smooth curve connecting $(0,0)$ with $(\lambda_\ast,
||u_{\lambda_\ast}||_{L^\infty(\Omega)})$, which is unbounded in the $||\cdot
||_{L^\infty}$ direction if $n\geq 10$.


% !TeX root = ../main.tex

\subsection{The case $\Omega=B^n$}
\label{sec:ball}
The celebrated theorem of Gidas-Ni-Nirenberg (see \cite{gidas-ni-nirenberg})
establishes that the solutions of
\begin{equation}
    (\mbox{GNN})\;\left\{ 
        \begin{array}{cc} \Delta u + f(u)=0 & \mbox{ in }B^n,\\
        u=0 & \mbox{ on }\partial B^n,
    \end{array}\right.
\end{equation}
are positive and radially symmetric provided that $f$ is positive and regular,
where $B^n=\{x\in\RR^n,|x|\leq 1\}.$ Therefore, if $\Omega=B^n$, all solutions
of $(G_\lambda)$ are radially symmetric and $(G_\lambda)$ is equivalent to the
problem of finding $u:[0,1]\to\RR$ such that
\begin{equation}
    \left\{\begin{array}{lc} 
            u''+\frac{n-1}{r}u'+\lambda\, e^u =0, & r\in (0,1),\\
    u'(0) = u(1) = 0.&  \end{array}\right.
\end{equation}
Notice that $||u||_{L^\infty}=u(0)$. For $n=1$, the equation $u''+\lambda e^u=0$
may be explicitely solved by means of the Laplace transform (see \cite{bratu}),
giving $u(x)=2\,\log\big(\alpha\,\mbox{sech}(\alpha\sqrt{\lambda/2}\cdot
x)\big)$, and the boundary conditions require that $\alpha$ is a solution to
the transcendental equation $\alpha
=\mbox{cosh}\big(\alpha\sqrt{\lambda/2}\big)$. One verifies that this leads to
zero, one, or two solutions for $\lambda>\lambda_c,\lambda=\lambda_c$ and
$\lambda<\lambda_c$ respectively, where $\lambda_c\approx .88$. Observe that
$||u||_{L^\infty}=u(0)=2\,\log\alpha$, and from the transcendental equation one
computes $\alpha_1=1+\frac{\lambda}{4}+o(1)$, $\alpha_2=\frac{4}{\lambda}+o(1)$.
It thus follows that, for any small and fixed $\lambda$, one solution
approaches to $||u||=0$ and the other one approaches to $||u||=\infty$, this is,
the solution curve $(\lambda, ||u_\lambda||)$ is unbounded, contained in
$[0,\lambda_c]\times \RR^+$. One can easily check that, in the upper branch of
solutions, $\lambda$ decreases with $||u||$, yielding the nose-shape of the
solution curve.

Other branches of solutions have been computed by numerical means, see for
instance \cite{new-solutions}.  The bi-dimensional case $\Omega=B^2$ can also
be solved explicitly and it presents a similar behaviour than the previous one.
Solutions exist if and only if $0\leq \lambda \leq 2$, and for $\lambda=2$
there is only one solution given by $u_\ast(r)=\log\frac{4}{(1+r^2)^2}$.
For other admissible values of $\lambda$, solutions are given by
$$
u_i(r)=\log\frac{b_i}{(1+(\lambda\, b_i/8)r^2)^2},
$$
where
$b_i=\frac{32}{\lambda^2}\big(1-\frac{\lambda}{4}+(-1)^i\sqrt{1-\lambda/2}\big),
i=1,2$.
\medskip

There is, however, a remarkable difference between $n=1$ and $n=2$: For
$\Omega=(-1,1)$, the unstable solution blows up at every point as $\lambda\to
0$, whereas for $\Omega=B^2$, it blows up only at the origin.
\medskip

For general dimension $n\geq 3$, the problem is analysed with a suitable change
of variables, and the behaviour of the solution yields from a dynamical coupled
system that arises from the study of the radial equation. This is the approach
considered by Dupaigne in \cite{stable-solutions-elliptic}; for other
instances of the Gelfand problem we strongly recommend this text. We summarize
the results for the radially symmetric case in every dimension in Table \ref{table:bifurcation-ball}.

\begin{table}[ht!]
    \center
    \renewcommand{\arraystretch}{1.2}
    \begin{tabular}{|c|p{.4\textwidth}|c|}
\hline
\textit{Dimension}                 & \textit{Number of solutions}                                            & \textit{Maximum value of $\lambda$}                  \\ \hline
\multirow{2}{*}{$n=1,2$}           & Two if $\lambda\in(0,\lambda_\ast)$                                     & \multirow{8}{*}{$\lambda_\ast(B^n)>2(n-2)$} \\ \cline{2-2}
                                   & One if $\lambda=\lambda_\ast$                                           &                                             \\ \cline{1-2}
\multirow{6}{*}{$3\leq n  \leq 9$} & One if $\lambda$ is sufficiently small                                  &                                             \\ \cline{2-2}
                                   & Finitely many if $\lambda\neq 2(n-2)$                                   &                                             \\ \cline{2-2}
                                   & More than any given number for $\lambda$ sufficiently close to $2(n-2)$ &                                             \\ \cline{2-2}
                                   & Infinitely many for $\lambda = 2(n-2)$                                  &                                             \\ \cline{2-2}
                                   & Two for $\lambda$ close to $\lambda_\ast(B^n)$                          &                                             \\ \cline{2-2}
                                   & One for $\lambda = \lambda_\ast(B^n)$                                   &                                             \\ \hline
$n\geq 10$                         & One unique stable solution                                              & $\lambda_\ast(B^n)=2(n-2)$                  \\ \hline
\end{tabular}
\caption{Bifurcation diagram for the Gelfand problem in $\Omega = B^n$ (\cite{stable-solutions-elliptic})}
\label{table:bifurcation-ball}
\end{table}

% !TeX root = ../main.tex

\subsection{Fermi coordinates}\label{sec:fermi}

Our framework is a specific case of the one used in~\cite{Pacard2014} for a
1-dimensional manifold, and we will indeed follow their notations. Consider the
tubular neighbourhood $T_\ep$ as defined in~\ref{eq-tube}; The Fermi coordinates
parameterise this set as a product space between the curve $\Lambda$ and
$B^{n-1}$ as follows. First identify $\Lambda$ with the zero-section of
$N\Lambda$ (the normal bundle of $\Lambda$) and $T_\ep$ with
\[
\Omega_\ep:=\{(y,z)\in N\Lambda;\, y\in\Omega,\, z\in N_y\Lambda,\, |z|\leq\ep\}
\]
via the natural mapping $T_\ep\to\Omega_\ep$, $(y,z)\mapsto y+z$.

If $g_z:=dz^2$ is the Euclidean metric on normal fibers, and $\mathring{g}$ is
the metric induced on $\Lambda$, we have that the metric $\bar g$ on $N\Lambda$
is induced by the embedding of $\Lambda$ in $\RR^n$, this is $\bar g=\mathring
g+g_z$. Lemma 3.2 of~\cite{Pacard2014} proves that, in these coordinates, the
Euclidean Laplacian $\Delta$ can be decomposed as
\begin{equation}
\Delta = \Delta_{\bar g} + D,
\end{equation}
where $\Delta_{\bar g}=\Delta_{\mathring g}+\Delta_{g_z}$ denotes the
Laplace-Beltrami operator on $N\Lambda$ for metric $\bar g$, and $D$ is a
second-order differential operator of the form
\begin{equation}
    D = \sum_{i=1}^{n-1} z_{i}D_i^{(2)}+D^{(1)},
\end{equation}
where $D^{(1)}$ and $D^{(2)}$ are first-order and second-order partial
differential operators respectively, whose
coefficients are smooth and bounded.
Note that in this $1$--dimensional manifold, we have a parameterisation
$t:\RR\to\Lambda$ of the curve, and the Laplace-Beltrami operator of $\Lambda$
is simply $\partial_{tt}$. We therefore have
\begin{equation}
    \Delta = \partial_{tt}+\Delta_{g_z}+D.
\end{equation}
This decomposition, and in particular the form of the operator $D$, will allow
us to obtain estimates for functions defined in small tubes and prove our main
results, which we describe in the next sections.


\section{The stable solution in the tube}
\label{sec:stable-sol-in-tube}

In this section, we establish the following:

\begin{theorem}
\label{thm:unique-stable-sol}
For small $\ep$, there exists a unique stable solution to
\begin{equation}
\label{eq:small-gelfand}
\left\{
\begin{array}{cc}
\Delta u + \frac{\lambda}{\ep^2}e^u=0 & \mbox{ in }T_\ep(\Lambda),\\
u = 0 & \mbox{ on }\partial T_\ep(\Lambda).
\end{array}
\right.
\end{equation}
\end{theorem}

\noindent An outline of the proof follows:

\begin{itemize}
\item Let $U$ be the unique radial stable solution of $(G_\lambda)$ for the
$(n-1)$-dimensional ball. We choose $u_\ep$, a rescaled version of $U$ that
travels along $\Lambda$, this is, for each section of the normal
bundle of $\Lambda$, consider the $(n-1)$-dimensional ball of radius $\ep$
centered about $\Lambda$ and use a copy of $U$.
\item The function $u_\ep$ verifies approximately the equation and it is stable
in the sense of section \ref{sec:stable-branch}. To prove this, we find a
super-solution to the problem and use a version of the maximum principle.
\item We use a fixed-point argument to prove the existence of a genuine solution
to \ref{eq:small-gelfand} of the form $u_\ep+v$, where $v$ is small. The
fixed-point theorem ensures the uniqueness of the perturbation if $v$ is small
enough. Finally, we show that larger perturbations do not lead to stable
solutions, proving the result.
\end{itemize}

Let us first collect some lemmas that will help us prove
\ref{thm:unique-stable-sol}.
\subsection{Some elliptic estimates}
Throughout this section, $||\cdot||$
stands for the $L^\infty$ norm in the tube
$T_\ep$, unless stated otherwise. Let $U(r)$ be the unique (radial) stable
solution of $(G_\lambda)$ for $\Omega
= B^{n-1}$, \ie
\begin{equation}
\left\{\begin{array}{cc}
\Delta U + \lambda\, e^U=0 & \mbox{ in }B^{n-1},\\
U = 0 & \mbox{ on }\partial B^{n-1},
\end{array}\right.
\end{equation}

\noindent and define $u_\ep:T_\ep\to \RR$ by
$$
u_\ep(y,z):=U\bigg(\frac{\mbox{dist}(z,\Lambda)}{\ep}\bigg).
$$
As the non-linearity $\lambda\,e^u$ is not multiplicative, we have that the
function $u_\ep$ will not verify the Gelfand equation in the tube, nor an
approximation, but the following estimate.

\begin{lemma}\label{lem-bound_Cepsilon} There is a constant $C$ such that
\begin{equation}
|\ep^2\Delta u_\ep+\lambda\, e^{u_\ep}|\leq C\ep.
\end{equation}
\end{lemma}

\begin{proof} It follows from the definition of $u_\ep$ that
$\ep^2\Delta_{g_z}u_\ep + \lambda\, e^{u_\ep}=0$, and note that $u_\ep$ does not
depend on the parameter $t$. Using the expression of the Euclidean Laplacian in
Fermi coordinates, we have
$$\ep^2\, \Delta u_\ep+\lambda e^{u_\ep} = (\ep^2\, \Delta_{g_z} u_\ep +
\lambda e^{u_\ep})+\ep^2(\partial_{tt}+D)u_\ep = \ep^2 D\, u_\ep,
$$
therefore, the estimate
\begin{eqnarray}
|\ep^2\Delta u_\ep + \lambda e^{u_\ep}|&\leq& \ep^2\, ||Du_\ep|| \\
&\leq& \sum_{i=1}^{n-1} \big(\ep^2||z_iD^{(2)}u_\ep||+\ep^2||D^{(1)}u_\ep||\big)\\
    &=& \sum_{i=1}^{n-1}\big(\ep
    ||x_iD^{(2)}U||_{L^\infty(B)}+\ep||D^{(1)}U||_{L^\infty(B)}\big)
\end{eqnarray}
holds and the result follows.
\end{proof}
\medskip

The stability of $U$ implies that there exist $\mu_1>0$ and $\phi_1>0$ such that
\begin{equation}
\left\{\begin{array}{cc}
-(\Delta+\lambda e^U)\phi_1=\mu_1\phi_1 & \mbox{ in }B^{n-1},\\
\phi=0 & \mbox{ on }\partial B^{n-1},
\end{array}\right.
\end{equation}
and hence the linearized operator about $U$ is invertible. This proves the
existence of a function $W$, which is a super-solution to the linearized
equation in $B^{n-1}$, verifying
\begin{equation}
\left\{\begin{array}{cc}
-(\Delta+\lambda e^U)W=1 & \mbox{ in }B^{n-1},\\
W=0 & \mbox{ on }\partial B^{n-1}.
\end{array}\right.
\end{equation}

\begin{lemma} There is a constant $B$ such that $|W|\leq B, |\nabla W|\leq
B, |\Delta W|\leq B$.
\end{lemma}
\begin{proof}
By the GNN theorem, $W$ is radially symmetric, positive and $\partial_r W<0$,
which gives $|W|\leq W(0)$. Write $W(z)=a(r)$, then $a$ solves
\begin{equation}
\left\{
\begin{array}{lc}
a''(r)+\frac{n-1}{r^2}a'(r)+\lambda e^Ua+1=0, & 0<r<1,\\
a(1) = 0, a'(0) = 0. &
\end{array}
\right.
\end{equation}
As $a(r)$ is bounded and positive, we have 
\begin{equation}
    \label{eq:bounding-a'}
-C\leq a''(r)+\frac{n-1}{r^2}a'(r)\leq -1
\end{equation}
for a positive constant $C$.
Multiplying \ref{eq:bounding-a'} by $e^{-\frac{n-1}{r}}$ we have $
-Ce^{-\frac{n-1}{r}}\leq \big(e^{-\frac{n-1}{r}}a'(r)\big)'\leq
e^{-\frac{n-1}{r}}$, \ie  
\begin{eqnarray*}
    |a'(r)|\leq  C_1\cdot e^{\frac{n-1}{r}}\int_0^re^{-\frac{n-1}{t}}\, dt \leq
    C_1  
\end{eqnarray*}
for a constant $C_1=\max(C,1)$. This yields $|\nabla W(z)|=|a(r)|\leq C_1$.
Finally, write $|\Delta W|\leq 1+e^UW\leq C_2$ for a positive constant $C_2$
and define $B:=\max(C_1,C_2,W(0))$.

\end{proof}

Similarly, define $w_\ep:T_\ep\to \RR$ by
$w_\ep(y,z):=W\big(\frac{\operatorname{dist}(z,\Lambda)}{\ep}\big)$. The
corresponding estimate for $w_\ep$ will be given by the following. 

\begin{lemma}
For sufficiently small $\ep$, the function $w_\ep$ verifies
\begin{equation}
    \label{eq-lemma--1/2}
\ep^2 \Delta w_\ep + \lambda e^{u_\ep}w_\ep \leq -1/2.
\end{equation}
\end{lemma}

\begin{proof}
Again note that $w_\ep$ does not depend on the parameter $t$ and that
$-(\ep^2\Delta_{g_z}w_\ep + e^{u_\ep}w_\ep)=1$. We then have that $(\ep^2\Delta
+ \lambda e^{u_\ep})w_\ep = (\ep^2\Delta_{g_z}+\lambda
e^{u_\ep})w_\ep+\ep^2(\partial_{tt} + D)w_\ep = -1+\ep^2Dw_\ep$. As $|\ep^2
Dw_\ep|\leq \ep C'$ for a constant $C'$ (as in the proof of lemma 1)
, the result follows for small $\ep$.
\end{proof}

Consider now the invertibility problem for the linearized operator
$L_\ep:=-(\ep^2 \Delta + e^{u_\ep})$ in $T_\ep(\Lambda)$: Given a smooth $f$ in
the tube, find $\phi$ such that
\begin{equation}
    \label{eq-lin-ep}
    \left\{
    \begin{array}{cc}
        L_\ep\phi = f & \mbox{ in }T_\ep(\Lambda),\\
        \phi = 0& \mbox{ on } \partial T_\ep(\Lambda).
    \end{array}
    \right.
\end{equation}

\begin{lemma} Let $\phi$ solve \ref{eq-lin-ep}, then
    \label{lem-phi}
    $\phi\leq 2 ||f||_{L^\infty(T_\ep)}\cdot w_\ep. $
    Moreover, $L_\ep$ is invertible given that $\ep$ is small enough.
\end{lemma}

\begin{proof} Let us first give the a-priori estimate by means of the maximum
    principle. From equations \ref{eq-lemma--1/2} and \ref{eq-lin-ep} we have
    \begin{equation}
        L_\ep w_\ep \geq \frac{1}{2}\;\mbox{ and }\;
        L_\ep\bigg(\frac{\phi}{2||f||}\bigg)\leq \frac{1}{2}.
    \end{equation}
    By addition, we have $L_\ep\big(w_\ep-\frac{1}{2||f||}\phi\big)\geq 0$.
    Recall the strong maximum principle: If $u$ is a smooth function verifying
    $-\Delta u\geq 0$ in a connected domain $\Omega$ of $\RR^n$, then if $u$
    attains a minimum in $\Omega$, $u$ is constant. The same conclusion holds
    if $-\Delta$ is replaced by $-\Delta+a(x)$ with $a\in L^p(\Omega)$ and
    $p\leq n/2$ (see \cite{1980JPhA...13..417H} for a proof of this theorem with
    applications to a Helium-like system). Note that the conclusion holds
    regardless of the sign of $a(x)$. Apply the strong maximum principle to the
    operator $L_\ep$ to conclude that either $w_\ep-\frac{1}{2||f||}\phi$ is
    constant or it attains a minimum in $\partial T_\ep(\Lambda)$. Because both
    $w_\ep$ and $\phi$ vanish on the boundary, in both cases
    $w_\ep-\frac{1}{2||f||}\phi\geq 0$, proving the first assertion.
    To prove that $L_\ep$ is invertible, write the
    problem in Fermi coordinates $(y,z)\in\Lambda\times B^{n-1}$:
    \begin{equation}
        \left\{
            \begin{array}{cc}
                -(\ep^2 \Delta_{\bar g}+\lambda e^{u_\ep})\phi - \ep^2\, D=f & \mbox{ in }T_\ep(\Lambda),\\
                \phi=0 & \mbox{ on }\partial T_\ep(\Lambda).
            \end{array}
            \right.
    \end{equation}

    Recall that $D$ is of the form $\sum_{i=1}^{n-1} z_iD^{(2)}+D^{(1)}$ where
    $D^{(i)}$ is an $i$--differential operator with smoothly bounded
    coefficients. In order to work in a domain with no dependence in $\ep$, use the
    scaling $z\mapsto z/\ep$ and define $\Phi(y,z):=\phi(y,\ep z),
    F(y,z):=f(y,\ep z)$.
    After simple manipulation, the problem reads
    \begin{equation}
        \left\{
            \begin{array}{cc}
                (L-\ep D)\Phi = F& \mbox{ in }T_1(\Lambda),\\
                \Phi=0 & \mbox{ on }\partial T_1(\Lambda).
            \end{array}
            \right.
    \end{equation}
    where $L=-(\Delta_{\bar g} + \lambda e^U)$ is invertible by hypothesis.
    Indeed, its eigenvalues are numbers of the form $\mu_i+\ep^2 \nu_j$ where
    $0<\mu_1\leq \mu_2\leq \mu_3\leq \dots$ are the eigenvalues of
    $-(\Delta_{g_z}+\lambda e^U)$ in $B^{n-1}$ and $0=\nu_1<\nu_2\leq \nu_3\leq
    \dots$ are the eigenvalues of $-\Delta_{\mathring g}$ on $\Lambda$.
    Therefore, $L-\ep D$ is a small perturbation of $L$, and we conclude that
    it is invertible for small $\ep$ after a simple fixed-point argument:
    Write $\Phi+L^{-1}F+\Psi$, then $\Psi$ solves 
    \begin{equation}
        \left\{
            \begin{array}{cc}
                \Psi = \ep L^{-1}D(L^{-1}F+\Psi)& \mbox{ in }T_1(\Lambda),\\
                \Psi=0 & \mbox{ on }\partial T_1(\Lambda).
            \end{array}
            \right.
    \end{equation}
    Using the fact that the norm of the inverse of $L$ is controlled by
    $1/\mu_1$, it is straightforward to show that the right-hand operator (to
    which we associate the $||\cdot||_\infty$ norm in the tube) is a
    contraction mapping for small $\ep$ that maps the space of bounded
    functions in the tube into itself. Thus $\Psi$ exists, allowing to conclude.
\end{proof}


\subsection{Proof of theorem \ref{thm:unique-stable-sol}}

\textit{Proof:} Write a perturbation of the solution $u=u_\ep+v$. Problem
    \ref{eq:small-gelfand} reduces to finding $v$ such that
    \begin{equation}
        \label{eq-fixed-point-H}
        \left\{
            \begin{array}{cc}
                \Delta (u_\ep+v)+\frac{\lambda}{\ep^2}e^{u_\ep+v}=0& \mbox{ in }T_\ep(\Lambda),\\
                v=0 & \mbox{ on }\partial T_\ep(\Lambda).
            \end{array}
            \right.
    \end{equation}
    Rewrite the differential equation as $(\ep^2\Delta + \lambda
    e^{u_\ep})v+(\ep^2\Delta u_\ep + \lambda e^{u_\ep})+\lambda
    e^{u_\ep}(e^v-1-v)=0$, and, recalling that $L_\ep$ is invertible for
    small $\ep$, define the following operator
    \begin{equation}
        H_\ep:= v\mapsto L_\ep^{-1}((\ep^2\Delta u_\ep + \lambda
        e^{u_\ep})+\lambda e^{u_\ep}(e^v-1-v)).
    \end{equation}
    Thus, any solution of \ref{eq-fixed-point-H}  is a fixed point of $H_\ep$. Let us introduce the space of functions
    $$\AAA_\ep :=\{v\in L^\infty(T_\ep(\Lambda)),\exists\, C \in \RR, ||v||\leq
    C\ep\},$$
    to which we associate the $L^\infty$ norm in the tube.
    % {\color{red}The space is not well defined because of the $\exists C$.}
    The following lemmas show that $H_\ep$ is a contraction mapping in
    $\AAA_\ep$. First, the fact that $H_\ep(\AAA_\ep)\subset \AAA_\ep$ for
    sufficiently small $\ep$ is a direct consequence of Lemma \ref{lem-phi} for
    a function in $\AAA_\ep$. 
    \begin{lemma}
        Let $v$ verify \ref{eq-fixed-point-H} and define $B=\max(1,2||\lambda
        e^U W||_{L^\infty(B^{n-1})})$. If $|v|\leq 1/B$, then $|v|\leq C\ep
        w_\ep$, where $C$ is a constant close to the constant from Lemma
        \ref{lem-bound_Cepsilon}.
    \end{lemma}
    \begin{proof} Lemma \ref{lem-phi}, applied for $v$ and $f=(\ep^2\Delta
        u_\ep+\lambda e^{u_\ep})+ \lambda e^{u_\ep}(e^v-1-v)$ gives
    \begin{eqnarray}
        v &\leq & 2 ||(\ep^2\Delta u_\ep + \lambda e^{u_\ep})+\lambda
        e^{u_\ep}(e^v-1-v)||w_\ep\\
          &\leq & C\ep w_\ep + B(e^v-1-v),
    \end{eqnarray}
    for a constant $C$ from Lemma \ref{lem-bound_Cepsilon} not depending on
    $\ep$. Using the fact that $e^t-1-t<t^2$ for $t\in[-1,1]$ (the first
    non-zero root of $g(t):=t^2-(e^t-1-t)$ is greater than $\ln(3)$), we
    conclude that, as $|v|\leq 1$, $|v|\leq C\ep w_\ep+Bv^2$. Thus,
    \begin{equation}
        \big(|v|-C\ep w_\ep+O(\ep^2)\big)\cdot\big( |v| -\frac{1}{B}-C\ep
        w_\ep+O(\ep^2)\big)\geq 0,
    \end{equation}
    from where the result follows. 
\end{proof}

\begin{lemma}
    $H_\ep$ is a contraction mapping of $\AAA_\ep$.
\end{lemma}
\begin{proof}
    Take $f,g\in \AAA_\ep$, then,
    $$||H_\ep(f-g)||=||L_\ep^{-1}\big(\lambda e^{u_\ep}(e^f-e^g -
    (f-g))\big)||.  $$
    The convexity of the exponential function implies that for any reals
    $\alpha,\beta$ with $\alpha>\beta$ we have $e^{\alpha}-e^{\beta}\leq
    e^{\alpha}(\alpha-\beta)$, and, taking $\beta=0$, $e^\alpha-1\leq \alpha e^\alpha$, therefore,
    \begin{eqnarray*}
        ||H_\ep(f-g)||&\leq& C\lambda ||e^{u_\ep}||\cdot ||e^{\max(f,g)}-1||\cdot
    ||f-g||\\
                      &\leq& C'\ep||f-g||,
    \end{eqnarray*}
    for a constant $C'$ not depending on $\ep$.
\end{proof}
This proves that there exists a
    unique genuine solution of $\ref{eq:small-gelfand}$ of the form $u_\ep+v$
    with $v$ small. The stability of this solution comes from the fact that the
    spectrum of the operators $-(\Delta+\lambda e^U)$ in $T_1(\Omega)$ and
    $-(\Delta + \lambda e^{u_\ep}$ in $T_\ep(\Omega)$ are equal: We have
    $$-(\Delta + \lambda e^{u_\ep+v})=-(\Delta + \lambda e^{u_\ep})+O(\ep)
    \mbox{ in } T_\ep(\Omega),$$
    and therefore the eigenvalues of the left-hand operator are close to the
    sequence $\mu_i$, which does not depend on $\ep$. In particular, they form
    a sequence of positive values for small $\ep$, which implies precisely that
    the associated quadratic form is positive definite. This allows to conclude
    that, for sufficiently small $\ep$, there exists a unique small perturbation
    that leads to a genuine stable solution of \ref{eq:small-gelfand}. This
    uniqueness holds only in a neighbourhood of $u_\ep$; it remains the
    question of whether there are stable solutions far from $u_\ep$. We answer
    negatively with the following argument: Suppose that there are two distinct
    stable solutions of \ref{eq:small-gelfand}. Their difference $v:=u_2-u_1$
    verifies
    $$-\ep^2 \Delta v = \lambda(e^{u_2}-e^{u_1}).$$
    Multiplying the above equation by the positive part of $v$ and integrating
    in the tube gives
    \begin{equation}
        \label{eq:unique-stable}
        \ep^2\int_{T_\ep}|\nabla v_+|^2\,dx = \lambda
    \int_{T_\ep}(e^{u_2}-e^{u_1})\cdot v_+\, dx.
\end{equation}
    As $u_2$ is stable, we have $\ep^2\int_{T_\ep}|\nabla v_+|^2\geq \lambda
    \int_{T_\ep} e^{u_2}\cdot v_+^2\, dx$. Plugging this inequality into
    \ref{eq:unique-stable} yields
    $$
    0 \leq \lambda \int_{T_\ep}(e^{u_2}-e^{u_1}-e^{u_2}v_+)v_+\, dx.
    $$
    Note that the integrand is a negative number by strict convexity of the
    exponential function, therefore, $v_+=0$. Changing $u_1$ and $u_2$ gives
    $v_-=0$, completing the proof of Theorem \ref{thm:unique-stable-sol}.
    \hfill $\blacksquare$

% \section{Unstable solutions in $\RR^2$ and $\RR^3$}

In this section, we show that the preceding construction generalises to obtain
new solutions to the Liouville problem in the tube given an unstable solution to
the problem in the ball $B^{n-1}$. We know that, for $n\geq 11$, there is a
unique solution to the Liouville problem in $B^{n-1}$, which is stable, thus we
will not be led to new solutions. We therefore work in tubes embedded in
$\RR^n$ with $n=2,3$.

The key ingredient that allowed us to prove the existence of the stable solution
in the tube is the invertibility of the operator $L_\ep=-(\Delta_{\mathring
g}+\lambda e^{u_\ep})$, where we recall that $u_\ep$ is a copy of the solution
given by the Gidas-Ni-Nirenberg theorem in the unit ball. Assume $2\leq n\leq 3$
and $\Omega=B^{n-1}$. We know from Section~\ref{sec:ball} that, for every
$\lambda < \lambda^\ast(B^{n-1})$, there is only one solution other than the
stable one, which we note by $\tilde U$. In other words, $\tilde U$ solves
\begin{equation}
\label{eq:u-tilde}
\left\{
\begin{array}{cc}
\Delta \tilde U + \lambda e^{\tilde U}=0 & \mbox{ in }B^{n-1},\\
\tilde U = 0 &\mbox{ on }\partial B^{n-1}.
\end{array}
\right.
\end{equation}

It is well-known that
$$
\mu_1(-\Delta-\lambda^\ast e^{u_\ast},B^{n-1})=0,
$$
where $u_\ast$ is the solution to $(L_\lambda^\ast)$ in the ball and $\mu_1$
stands for the least eigenvalue (see~\cite{stable-solutions-elliptic}, \S 2),
is the only case where the operator $-\Delta-\lambda^\ast e^{u_\ast}$ is
non-invertible for these dimensions. In this case, let us note, for any
$\lambda < \lambda^\ast$,
$$\mu_1<0<\mu_2\leq \mu_3\leq \cdots$$
the eigenvalues of the operator $-\Delta-\lambda e^{\tilde U}$ in the unit ball
$B^{n-1}$, and define as before $\tilde u_\ep:T_\ep\to\RR$ by $\tilde
u_\ep(y,z)=\tilde U\big(\frac{\operatorname{dist}(z,\Lambda)}{\ep}\big)$. The
negative eigenvalue $\mu_1$ is the source of a resonance phenomenon, forcing
to tweak part of the argument. Let us explain briefly: Note
$$0=\nu_1<\nu_2\leq\nu_3\leq\dots $$
the eigenvalues of $-\Delta_{\mathring g}$ on $\Lambda$. If we proceed as
before, the argument fails when trying to invert $\tilde
L_\ep:=-\ep^2\Delta+\lambda e^{\tilde u_\ep}$. Indeed, with a rescaling to work
in $T_1( \Lambda )$, this operator becomes $(-\Delta_{g_z}-\ep^2
\Delta_{\mathring g}-\ep D):=\hat L - \ep D$. Now, the eigenvalues of $\hat L$
are of the form $\mu_i+\ep^2\nu_j$, and as $\mu<0$, there exist values of $\ep$
such that $\hat L$ is non invertible. Let us note
$$
S:=\{\ep>0\,:\; \mu_1+\ep^2\nu_j=0,\, j\in \NN\}
$$
the set of such values of $\ep$. Even if we restrict $\ep\not\in S$, we will not
conclude at once the invertibility of $\tilde L_\ep$, because the fixed-point
theorem uses the fact that the norm of the inverse is controlled by
$1/\delta_\ep$, where $\delta_\ep$ is the distance from $0$ to the spectrum,
which was $\mu_1$ before. Here, we only have the bound
$$
\delta_\ep\leq\ep^2 \max_{1\leq j \leq k}(\nu_{j+1}-\nu_j),
$$
where $k$ is the index of the first eigenvalue of the spectrum
$(\nu_j)_{j\in\NN}$ such that $\mu_1+\ep^2\nu_k>0$. In particular, we will not
find a contraction mapping leading to the invertibility of $\tilde L_\ep$.

Let us first prove a partial result, and concentrate in $\tilde L_\ep$ afterwards.

\begin{theorem}
    Let $n=2,3$. For small enough $\ep$ and such that $\tilde L_\ep$ is
    invertible, there exist two solutions to
    \begin{equation}
        \label{eq-thm-2-sols}
        \left\{
            \begin{array}{cc}
                \Delta u + \frac{\lambda}{\ep^2} e^u = 0 & \mbox{ in
                }T_\ep(\Lambda),\\
                    u=0 & \mbox{ on }\partial T_\ep(\Lambda).
                \end{array}
                \right.
    \end{equation}
    \end{theorem}
    \textit{Proof: } Define by analogy the functions $\tilde W, \tilde w_\ep$
    as supersolutions and prove the same estimates as before. The fact that
    $\tilde L_\ep$ is invertible ensures that we have elliptic estimates for
    $\tilde u_\ep, \tilde w_\ep$. To finish the construction, restate the
    problem~\ref{eq-thm-2-sols} as a fixed-point problem that can be easily
    solved as in the proof of theorem~\ref{thm:unique-stable-sol}, completing
    the proof.\hfill $\blacksquare$

    % !TeX root = ../main.tex

\subsection{Morse index of $\tilde L_\ep$}

The spectral analysis of $\tilde L_\ep$ follows directly from the ideas
of~\cite{Pacard2014}, where authors deal with a power non-linearity. We adapt
the same estimates here. Recall that the normal bundle $N\Lambda$ is endowed
with $\bar g= \mathring g + g_z$ in $\RR^n$, where $\mathring g$ is the induced
metric on $\Lambda$ and $g_z:=dz^2$ is the Euclidean metric on normal fibers of
$N\Lambda$. We deal with the operators
\[
    \hat L_\ep:=-(\ep^2\Delta_{\bar g}+\lambda e^{\tilde u_\ep}) \quad \mbox{and}
    \quad \tilde L_\ep := -(\ep^2\Delta + \lambda e^{u_\ep}),
\]
to which we associate the quadratic forms
\begin{eqnarray*}
    \hat Q(v) &=& \ep^2 \int_{T_\ep}|\nabla_{\bar g} v|^2\, \operatorname{dvol}_{\bar g}
    - \lambda \int_{T_\ep}e^{u_\ep}v^2 \, \operatorname{dvol}_{\bar g}\\
    \tilde Q(v) &=& \ep^2 \int_{T_\ep}|\nabla_{g} v|^2\, \operatorname{dvol}_{g}
    - \lambda \int_{T_\ep}e^{u_\ep}v^2 \, \operatorname{dvol}_{g}\\
\end{eqnarray*}
respectively, where $g$ stands for the Euclidean metric.

We know explicit eigenvalues of $\hat L_\ep$, which, together with Weyl's
approximation formula, will give us an estimate of the Morse index of
$\tilde L_\ep$.

\begin{lemma}
    There exists a constant $C>0$ such that, for all sufficiently small $\ep>0$,
    \[
        \operatorname{Index}(\tilde L_\ep)\leq \frac{C}{\ep}.
    \]
\end{lemma}

\begin{proof} Let $w\in H_0^1(T_\ep)$ with $||w||_{L^2(T_\ep)}=1$ and
    satisfying $\tilde Q(w)\leq 0$. We will give an estimate for $\hat Q(w)$.
    First,
    \begin{eqnarray}
        \label{eq-nabla-w}
        ||\nabla_g w|^2-|\nabla_{\bar g} w|^2|&\leq& C\ep |\nabla_g w|^2,\\
        |\sqrt{\det g} - \sqrt{\det \bar g}|&\leq& C\ep\sqrt{\det \bar g}
    \end{eqnarray}
    follow at once from the fact that, if one considers a parameterisation $\Phi$
    from a neighbourhood of $(y,0)\in \Lambda\times \RR^{n-1}$ into a
    neighbourhood of $y\in\RR^n$ by
    \[
        \Phi(y,z_1,\dots,z_{n-1}):=y + \sum_{i=1}^{n-1}z_i e^i(y),
    \]
    where $\{e^1,\dots, e^{n-1}\}\subset N\Lambda$ is a moving orthonormal
    frame in which each $e^j$ is a smooth section of the normal bundle of
    $N\Lambda$, one can show that the pull-back of $g$ by $\Phi$ is close
    to $\bar g$. For completeness, let us give a quantitative version of this
    statement, which is proved in~\cite{Pacard2014} (Lemma 3.1). {\color{red}
    Expresar referencia como lema.} Using~\ref{eq-nabla-w}, we have
    \begin{eqnarray*}
        \hat Q(w) &\leq & \hat Q(w)-\tilde Q(w)\\
                  &\leq & C\ep\bigg(\int_{T_\ep} \ep^2 \cdot |\, |\nabla_g
                      w|^2-|\nabla_{\bar g} w|^2|\, d\mbox{vol}_{\bar g}
                      + \int_{T_\ep} \lambda e^{u_\ep}w^2\,
                  d\operatorname{vol}_{\bar g}\bigg)\\
                  &\leq & C\ep\int_{T_\ep}(\ep^2|\nabla_g w|^2+\lambda
                  e^{u_\ep}w^2) \, d\mbox{vol}_g.
    \end{eqnarray*}
    Also, $\tilde Q(w)\leq 0$ implies
    \begin{equation}
        \int_{T_{\ep}} \ep^2|\nabla_g w|^2\, d\mbox{vol}_g \leq \lambda
        \int_{T_\ep}e^{u_\ep}w^2\, d\mbox{vol}_g \leq C'\int_{T_\ep}w^2\,
        d\mbox{vol}_{\bar g}=C',
    \end{equation}
    from where we conclude $\hat Q(w)\leq C\ep$ for a posivite constant $C$.
    This means that the index of $\tilde L_\ep$ is less than the number of
    independent eigenfunctions of $\hat L_\ep$ ssociated to eigenvalues that
    do not exceed $C\ep$. In other words,
    \begin{eqnarray*}
        \mbox{Index}(\hat L_\ep) &\leq & \#\{j\in\NN:\, \mu_i+\ep^2\nu_j\leq
        C\ep\}\\
                                 & = & \#\left\{j\in\NN:\,\nu_j\leq
                                 \frac{C}{\ep}-\frac{\mu_i}{\ep^2}\right\}\\
                                 & \leq & \#\left\{j\in\NN:\,\nu_j\leq
                                 \frac{C}{\ep}-\frac{\mu_1}{\ep^2}\right\}\\
                                 & \leq & \#\left\{j\in\NN:\,\nu_j\leq
                                     \frac{\alpha}{\ep}
                                 \right\}\approx \frac{1}{\ep},
        \end{eqnarray*}
        where the last estimate uses the fact that, from Weyl's formula, the number
        of eigenvalues of $-|\Delta_{\mathring g}|$ counted with multiplicity which
        are less than $\theta>0$ is equivalent to $\sqrt{\theta}$ as $\theta\to\infty$.

    \end{proof}

    
\subsection{Exploiting Pacard-Pacella-Sciunzi's results}\label{sec:exploiting-pacard}

In section 5 of~\cite{Pacard2014}, authors prove that any operator of the form
$-(\Delta+V(x))$ is invertible in the tube, starting from the invertibility of
$-(\frac{1}{\ep^2}\Delta_{g_z}+\Delta_{\mathring g}+V(x))$ in the same domain.
Their analysis for the power non-linearity can be applied by analogy with no
further complications to obtain the invertibility of $\tilde L_\ep$.

Let us describe the decomposition of eigenfunctions of $\tilde L_\ep$. Let
$\Phi_1:B^{n-1}\to \RR$ denote the first eigenfunction of $-(\Delta+\lambda
e^{\tilde U})$ in the ball $B^{n-1}$ with Dirichlet boundary conditions. Using
separation of variables we have that the eigenfunction of $\hat L_\ep$
associated to the eigenvalue $\mu_1+\ep^2\nu_j$ can be decomposed as
$\phi_1(z)\psi_j(y)$, where $\psi_j$ is the $j$-th eigenfunction of
$-\ep^2\Delta_{\mathring g}$ in $\Lambda$ and $\phi_1(z):=\Phi_1(z/|\ep|)$. Here
we prove a similar property for the operator $\tilde L_\ep$. We insist on the
fact that the ideas and lemmas here are a direct adaptation of the power
non-linearity treated in~\ref{sec:exploiting-pacard}. First, for fixed $\ep$
define the smooth function $a$ and the operator $\tilde H_\ep$ acting on
functions in the tube by
\begin{equation}
    \operatorname{dvol}_g = \operatorname{dvol}_{\bar g}\quad\;\mbox{and}\;
    \quad \tilde H_\ep:= a\tilde L_\ep,
\end{equation}
so that $\tilde L_\ep$ is self-adjoint with respect to $L^2(T_\ep(\Lambda),g)$
and $\tilde H_\ep$ is self-adjoint with respect to $L^2(T_\ep(\Lambda),
\bar{g})$. We then have the following.

\begin{lemma}
    Let $v$ be an eigenfunction of $\tilde H_\ep$ associated to the eigenvalue
    $\gamma$, and for a function $\psi$ defined on $\Lambda$ write
    \begin{equation}
        v(y,z):=\phi_1(z)\psi(y)+w(y,z)
    \end{equation}
    with the orthogonality condition
    \begin{equation}
        \forall h \in L^2(\Lambda),\; \int_{T_\ep}w\cdot \phi_1\cdot h\, \operatorname{dvol}_{\bar g} = 0.
    \end{equation}
    Then there exists a constant $C$ such that
    \begin{equation}
        \int_{T_\ep}(|\nabla_{\bar g}w|^2+w^2)\operatorname{dvol}_{\bar g} \leq
        C\gamma\ep\int_{T_\ep} v^2\operatorname{dvol}_{\bar g}.
    \end{equation}
\end{lemma}
We omit the proof as it follows by analogy from the proof of Lemma 5.2
in~\cite{Pacard2014}. Remark, however, that this lemma is more accurate than
its
analogue, because we do not need assumptions on the eigenvalue $\gamma$ and
because the estimation is given in term of a positive power of $\ep$. In
particular, we can see the function $w$ as a small perturbation of $v$ in the
$L^2(T_\ep(\Lambda))$ sense.
\medskip

Pacard et.al.\ also give an estimate for the change rate of the eigenvalues as
$\ep$ increases, which we adapt here as well:

\begin{lemma}\label{lem-partial-nu}
    If $\nu$ is an eigenvalue of $\tilde H_\ep$ such that $\nu<\alpha$, then
    there is a constant $C$ such that
    \[
        \frac{\partial \nu}{\partial \ep} \geq C\alpha\ep.
    \]
\end{lemma}
\begin{proof}
    {\color{red} To do. Paragraph of page 14 is suspicious.}
\end{proof}

    \subsection{Invertibility of $\tilde L_\ep$}

One may apply the preceding analysis to conclude the following.

\begin{theorem}
Given $N\geq 3$, there exists a set $A^N\subset (0,+\infty)$ and $N_0\in \NN$
such that, for all $\ep\in A^N$, the operator $\tilde L_\ep$ is invertible and
the norm of its inverse defined from $C^0(T_\ep(\Lambda))$ to
$C_0^1(T_\ep(\Lambda))$ (the subspace of $C^1(T_\ep(\Lambda))$ spanned by
functions that vanish on $\partial T_\ep(\Lambda)$) is bounded by a constant
times $\ep^{-1-N-N_0}$. Moreover, for $\alpha>N-1$,
\[
\lim_{\ep\to 0}\frac{\ep -\operatorname{meas}(A^N\cap(0,\ep))}{\ep^\alpha} = 0.
\]
\end{theorem}

\begin{proof}
    Fix $\ep>0$, denote by $\Sigma_\ep$ the spectrum of $\tilde H_\ep =
    a\tilde{L_\ep}$ and the set of ``resonant'' values of $\ep$ by
 \[
 R_\ep:=\{\ep>0:\, 0\in S_\ep\}
 \]
 {\color{red} Shouldnt it be $\Sigma$ instead of $S$?}

 It is easy to show that if $\ep\neq R_\ep$, the norm of the inverse of
 $\tilde{H_\ep}$ can be estimated by a constant times $1/\delta_\ep$ where
 $\delta:=\min\{|\mu|:\, \mu\in S_\ep\}$ {\color{red} same} is the distance
 from 0 to the spectrum.  Fix now $N\geq 2$ and define, for all $\ep\in(0,1)$,
 the subset of ${}^c(R_\ep)$
 \begin{equation}
    A_\ep^N:=\{\alpha\in(\ep,2\ep):\, (\alpha-\ep^N,\alpha + \ep^N)\cap
    R_\ep=\emptyset\},
\end{equation}
which consists of values that are far enough from the resonant ones. The fact
that the Morse index of $\tilde L_\ep$ is bounded by $C/\ep$ implies that
$\ep-\mbox{meas}(A_\ep^N)$ (the quantity of negative eigenvalues of $H_\ep$ up
to a multiplicative constant) cannot belarger than a constant times
$\ep^{N-1}$. The result of Lemma~\ref{lem-partial-nu} also implies that the
norm of the inverse of $\tilde H_\ep$ is bounded by a constant times
$\ep^{-1-N}$ and, as $\tilde L_\ep = a\tilde H_\ep$ where $a$ is bounded away
from 0, we have the same property for $L_\ep$. If we let
\begin{equation}
    A^N:=\bigcup_{\ep\in(0,1)} A_\ep^N,
\end{equation}
then for every $\ep\in A^n$, $\tilde L_\ep$ is invertible, when defined from
$L^2(T_\ep)$ to itself, and the norm of its inverse is bounded by a constant
times $\ep^{-1-N}$. The norm of the inverse of $\tilde{L}_\ep$ when defined
from $C^0(T_\ep)$ into $C_0^1(T_\ep)$ follows from Schauder's estimates:
indeed, up to some powers of $\ep$, say $N_0$, we have the same property. This
completes the proof.
\end{proof}



    We have therefore proven that, in dimensions 2 and 3, there exist two
    solutions to the problem given that $\ep$ is away from the set of values
    that lead to resonance phenomena: the stable one discussed
    in~\ref{sec:stable-sol-in-tube} and the unstable one, with Morse index
    $\approx 1/\ep$.

% \section{Unstable solutions in $\RR^4,\RR^5,\dots,\RR^{10}$}

We prove the following theorem, stating that the bifurcation diagram of the
Gelfand problem in the tube is similar to the one given by
\cite{stable-solutions-elliptic}, provided $\ep$ is away from a resonant
set.

\begin{theorem}
    Let $4\leq n \leq 10$. There is a set $E$ that accumulates around 0, such
    that if $\ep\in E$, the following holds: 
    \begin{itemize}
        \item For small $\lambda$ there is a unique, stable solution to
           ~\ref{eq-thm-2-sols}.
        \item For $\lambda$ near $\lambda^\ast(B^{n-1})$, there are two
            solutions to~\ref{eq-thm-2-sols}, a stable and an unstable one.
        \item For any given $k\in \NN$, there exists $\lambda$ near $2(n-3)$
            such that the number of solutions of~\ref{eq-thm-2-sols} is at least
            $k$. The Morse index of each of the unstable solutions converges to
            $\infty$ as $\ep\to 0$ in $E$.
    \end{itemize}
\end{theorem}

{\color{red} too informal...}

\section{Concluding remarks}

In this article, we adapted the arguments
of~\cite{Pacard2014,stable-solutions-elliptic} to the Gelfand problem $\Delta u
+ \lambda e^u=0$ with Dirichlet boundary conditions in a small geodesic tube
around a smooth manifold embedded in $\RR^n$.
% As already stated in~\cite{Pacard2014},
We believe that the same line of thinking in~\cite{Pacard2014} can be
carried out without further complications here to complete the bifurcation diagram.
This is, to show that the unstable
solutions of the Gelfand problem in the unit ball can also be used to construct
unstable solutions in tubular domains, for $\ep$ possibly outside a set of
resonant values in which the linearized operator $-\Delta - \lambda e^{u_\ep}$
is not invertible, where $u_\ep$ is the rescaling of an unstable solution. Moreover,
an analogous spectral analysis of the linearized operator should yield the
Morse index of unstable solutions. This way, we expect the same resonance phenomena,
i.e., that the invertibility of linearized operators holds only for values of $\ep$
in a set accumulating around 0 (but not for every small $\ep>0$).

\medskip

\textit{Acknowledgements: During this work, the author was guest
    researcher at the Facultad de Matem\'aticas de la Pontificia Universidad
    Cat\'olica de Chile. The author thanks Monica Musso and Frank Pacard.}

\bibliographystyle{alpha}
%\nocite{*}
\bibliography{bibliography/biblio}

% \appendix 

\end{document}
